\documentclass[12pt,dvipdfmx]{beamer}
\institute{}
\author{田浦健次朗}
\date{}

\usepackage{graphicx}
\DeclareGraphicsExtensions{.pdf}
\DeclareGraphicsExtensions{.eps}
\graphicspath{{out/}{out/tex/}{out/tex/gpl/}{out/tex/svg/}{out/tex/lsvg/}{out/tex/dot/}}
% \graphicspath{{out/}{out/tex/}{out/pdf/}{out/eps/}{out/tex/gpl/}{out/tex/svg/}{out/pdf/dot/}{out/pdf/gpl/}{out/pdf/img/}{out/pdf/odg/}{out/pdf/svg/}{out/eps/dot/}{out/eps/gpl/}{out/eps/img/}{out/eps/odg/}{out/eps/svg/}}
\usepackage{listings,jlisting}
\usepackage{fancybox}
\usepackage{hyperref}
\usepackage{color}

%%%%%%%%%%%%%%%%%%%%%%%%%%%
%%% themes
%%%%%%%%%%%%%%%%%%%%%%%%%%%
\usetheme{default} % Szeged
%% no navigation bar
% default boxes Bergen Boadilla Madrid Pittsburgh Rochester
%% tree-like navigation bar
% Antibes JuanLesPins Montpellier
%% toc sidebar
% Berkeley PaloAlto Goettingen Marburg Hannover Berlin Ilmenau Dresden Darmstadt Frankfurt Singapore Szeged
%% Section and Subsection Tables
% Copenhagen Luebeck Malmoe Warsaw

%%%%%%%%%%%%%%%%%%%%%%%%%%%
%%% innerthemes
%%%%%%%%%%%%%%%%%%%%%%%%%%%
% \useinnertheme{circles}	% default circles rectangles rounded inmargin

%%%%%%%%%%%%%%%%%%%%%%%%%%%
%%% outerthemes
%%%%%%%%%%%%%%%%%%%%%%%%%%%
% outertheme
% \useoutertheme{default}	% default infolines miniframes smoothbars sidebar sprit shadow tree smoothtree


%%%%%%%%%%%%%%%%%%%%%%%%%%%
%%% colorthemes
%%%%%%%%%%%%%%%%%%%%%%%%%%%
\usecolortheme{seahorse}
%% special purpose
% default structure sidebartab 
%% complete 
% albatross beetle crane dove fly seagull 
%% inner
% lily orchid rose
%% outer
% whale seahorse dolphin

%%%%%%%%%%%%%%%%%%%%%%%%%%%
%%% fontthemes
%%%%%%%%%%%%%%%%%%%%%%%%%%%
\usefonttheme{serif}  
% default professionalfonts serif structurebold structureitalicserif structuresmallcapsserif

%%%%%%%%%%%%%%%%%%%%%%%%%%%
%%% generally useful beamer settings
%%%%%%%%%%%%%%%%%%%%%%%%%%%
% 
\AtBeginDvi{\special{pdf:tounicode EUC-UCS2}}
% do not show navigation
\setbeamertemplate{navigation symbols}{}
% show page numbers
\setbeamertemplate{footline}[frame number]

%%%%%%%%%%%%%%%%%%%%%%%%%%%
%%% define some colors for convenience
%%%%%%%%%%%%%%%%%%%%%%%%%%%

\newcommand{\mido}[1]{{\color{green}#1}}
\newcommand{\mura}[1]{{\color{purple}#1}}
\newcommand{\ore}[1]{{\color{orange}#1}}
\newcommand{\ao}[1]{{\color{blue}#1}}
\newcommand{\aka}[1]{{\color{red}#1}}

\setbeamercolor{ex}{bg=cyan!20!white}

\definecolor{UniBlue}{RGB}{20,20,250} 
\setbeamercolor{structure}{fg=UniBlue} % 見出しカラー

\iffalse
%%%%%%%%%%%%%%%%%%%%%%%%%%%
%% customize beamer template
%% https://www.opt.mist.i.u-tokyo.ac.jp/~tasuku/beamer.html
%%%%%%%%%%%%%%%%%%%%%%%%%%%

%\renewcommand{\familydefault}{\sfdefault}  % 英文をサンセリフ体に
%\renewcommand{\kanjifamilydefault}{\gtdefault}  % 日本語をゴシック体に
\usefonttheme{structurebold} % タイトル部を太字
\setbeamerfont{alerted text}{series=\bfseries} % Alertを太字
\setbeamerfont{section in toc}{series=\mdseries} % 目次は太字にしない
\setbeamerfont{frametitle}{size=\Large} % フレームタイトル文字サイズ
\setbeamerfont{title}{size=\LARGE} % タイトル文字サイズ
\setbeamerfont{date}{size=\small}  % 日付文字サイズ

\definecolor{UniBlue}{RGB}{0,150,200} 
\definecolor{AlertOrange}{RGB}{255,76,0}
\definecolor{AlmostBlack}{RGB}{38,38,38}
\setbeamercolor{normal text}{fg=AlmostBlack}  % 本文カラー
\setbeamercolor{structure}{fg=UniBlue} % 見出しカラー
\setbeamercolor{block title}{fg=UniBlue!50!black} % ブロック部分タイトルカラー
\setbeamercolor{alerted text}{fg=AlertOrange} % \alert 文字カラー
\mode<beamer>{
    \definecolor{BackGroundGray}{RGB}{254,254,254}
    \setbeamercolor{background canvas}{bg=BackGroundGray} % スライドモードのみ背景をわずかにグレーにする
}


%フラットデザイン化
\setbeamertemplate{blocks}[rounded] % Blockの影を消す
\useinnertheme{circles} % 箇条書きをシンプルに
\setbeamertemplate{navigation symbols}{} % ナビゲーションシンボルを消す
\setbeamertemplate{footline}[frame number] % フッターはスライド番号のみ

%タイトルページ
\setbeamertemplate{title page}{%
    \vspace{2.5em}
    {\usebeamerfont{title} \usebeamercolor[fg]{title} \inserttitle \par}
    {\usebeamerfont{subtitle}\usebeamercolor[fg]{subtitle}\insertsubtitle \par}
    \vspace{1.5em}
    \begin{flushright}
        \usebeamerfont{author}\insertauthor\par
        \usebeamerfont{institute}\insertinstitute \par
        \vspace{3em}
        \usebeamerfont{date}\insertdate\par
        \usebeamercolor[fg]{titlegraphic}\inserttitlegraphic
    \end{flushright}
}
\fi

%%%%%%%%%%%%%%%%%%%%%%%%%%%
%%% how to typset code
%%%%%%%%%%%%%%%%%%%%%%%%%%%

\lstset{language = C,
numbers = left,
numberstyle = {\tiny \emph},
numbersep = 10pt,
breaklines = true,
breakindent = 40pt,
frame = tlRB,
frameround = ffft,
framesep = 3pt,
rulesep = 1pt,
rulecolor = {\color{blue}},
rulesepcolor = {\color{blue}},
flexiblecolumns = true,
keepspaces = true,
basicstyle = \ttfamily\scriptsize,
identifierstyle = ,
commentstyle = ,
stringstyle = ,
showstringspaces = false,
tabsize = 4,
escapechar=\@,
}

\AtBeginSection[]
{
\begin{frame}
\frametitle{}
\tableofcontents[currentsection]
\end{frame}
}

\AtBeginSubsection[]
{
\begin{frame}
\frametitle{}
\tableofcontents[currentsection,currentsubsection]
\end{frame}
}


\title{ユーザレベル仮想記憶APIとその応用}
\begin{document}
\maketitle

%%%%%%%%%%%%%%%%%%%%%%%%%%%%%%%%%% 
\begin{frame}
\frametitle{目次}
\tableofcontents
\end{frame}

%%%%%%%%%%%%%%%%% 
\section{ユーザレベル仮想記憶API}
%%%%%%%%%%%%%%%%% 

%%%%%%%%%%%%%%%%% 
\begin{frame}
  \frametitle{ユーザレベル仮想記憶APIとは}
  \begin{itemize}
  \item {\tt mprotect(\ao{\it a}, \ao{\it len}, \ao{\it prot});}
    \begin{itemize}
    \item アドレス範囲$[a, a+\mbox{\it len})$に保護属性{\it prot}を設定
    \end{itemize}
  \item {\tt char * {\it a} = mmap(\ao{\it a'}, \ao{\it len}, \ao{\it prot},
      {\it flags}, {\it fd}, {\it offs});}
    \begin{itemize}
    \item $[a, a+\mbox{\it len})$を割り当てるとともに,
      保護属性{\it prot}を設定
    \item $a' = 0 \Rightarrow a = $OSが探した空き領域
    \item $a' \neq 0 \Rightarrow$
      $[a, a+\mbox{\it len})$が空いていれば$a = a'$
    \end{itemize}
  \item {\it prot} --- {\tt PROT\_READ, PROT\_WRITE, PROT\_EXEC} (それらのbit和)
  \item {\tt sigaction(SIGSEGV, {\it act}, {\it oldact});}
    \begin{itemize}
    \item Segmentation Fault発生時に, {\it act}で指定したシグナルハンドラを実行
    \end{itemize}
  \end{itemize}
\end{frame}

%%%%%%%%%%%%%%%%% 
\begin{frame}
  \frametitle{mprotect/mmapの保護属性設定の実装}
  \begin{itemize}
  \item アドレス空間記述表に保護属性を記述し,
    \ao{必要に応じて}ページテーブルを設定するだけ
    \begin{itemize}
    \item \ao{必要に応じて} $\equiv$
      ページテーブルの設定で可能なアクセス $\subset$
      アドレス空間記述表で可能なアクセス
    \end{itemize}
  \item OSにとっては, もともと他の目的に使っていた道具(ページテーブル)を,
    ユーザに(も)使わせているだけ
  \item もともと割り当てた領域はREAD/WRITE許可だったのを,
    READ/WRITE禁止を可能にしているだけで, 安全性の問題もない
  \item むしろ「わざわざREAD/WRITE禁止にして何の役に立つのか」が疑問
  \end{itemize}
\end{frame}

%%%%%%%%%%%%%%%%% 
\begin{frame}
  \frametitle{なぜ役に立つのだろうか?}
  \begin{itemize}
  \item 思い出そう: OS自身によるページテーブルのうまい利用法
    \begin{itemize}
    \item \ao{要求時ページング}
    \item \ao{mmap} (e.g., 大きなファイルのランダムアクセス)
    \item \mura{コピーオンライト}による物理メモリの節約
    \item \mura{forkの高速化}(コピーオンライトの応用)
    \item プロセス間共有メモリ
    \end{itemize}
  \item 共通アイデア: \ao{ページフォルト},
    \mura{保護違反}の例外をキャッチして適切な処理を行う
  \item $\Rightarrow$
    \begin{itemize}
    \item ユーザに使わせればさらなる応用があるはず
    \item そもそも実装も簡単(ページテーブルを適切に設定するだけ)だし
    \end{itemize}
  \end{itemize}
\end{frame}

%%%%%%%%%%%%%%%%% 
\begin{frame}
  \frametitle{多くの応用に共通の基本アイデア}
  \begin{itemize}
  \item ある領域をREAD (WRITE)不可に設定
  \item 通常通りプログラムを実行
  \item 途中で保護違反が発生したら,その場所をREAD (WRITE)可に設定 $+$ 実行を継続
  \item ポイント
    \begin{itemize}
    \item 実行結果は通常と変わらない
    \item 「実行中どこ(どのページ)にアクセスしたか」の情報が得られる
    \end{itemize}
  \end{itemize}
\end{frame}


%%%%%%%%%%%%%%%%% 
\begin{frame}
  \frametitle{これまでに発明された応用}
  \begin{itemize}
  \item \{差分・並行\}チェックポインティング
  \item ネットワークページング, 分散共有メモリ
  \item インクリメンタル Garbage Collection
  \end{itemize}
  興味のある人は,
  \begin{itemize}
  \item  Andrew W. Appel and Kai Li. ``Virtual Memory Primitives for User Programs''
  \end{itemize}
\end{frame}


%%%%%%%%%%%%%%%%% 
\section{チェックポインティング}
%%%%%%%%%%%%%%%%% 

%%%%%%%%%%%%%%%%% 
\begin{frame}
  \frametitle{チェックポインティングとは}
  \begin{itemize}
  \item 後から再開できるよう, 計算の途中状態を(通常2次記憶に)保存すること
    \begin{itemize}
    \item 保存した状態のことを「チェックポイント」と言う
    \end{itemize}
  \item 主な目的: 耐故障性; 保存した状態から計算を再開
  \item 単純な方法:
    \begin{itemize}
    \item 必要なアプリケーションデータをすべて, 保存
    \end{itemize}

    \begin{center}
      \includegraphics[width=0.8\textwidth]{out/pdf/svg/checkpoint_1.pdf}      
    \end{center}

  \item 問題点:
    \begin{itemize}
    \item データが大きい場合に時間がかかる $\rightarrow$ \ao{差分}チェックポインティング
    \item その間アプリケーションが停止している
      $\rightarrow$ \ao{並行}チェックポインティング
    \end{itemize}
  \end{itemize}
\end{frame}

%%%%%%%%%%%%%%%%% 
\begin{frame}
  \frametitle{差分チェックポインティング}
  \begin{itemize}
  \item チェックポイント保存時, 前回との「差分」のみ保存
    \begin{itemize}
    \item 「差分」$=$ 前回のチェックポイント以降に変更($\approx$書き込み)があった部分
    \end{itemize}
  \item \ao{書き込まれた部分を知るために仮想記憶API}を使う
    \begin{enumerate}
    \item チェックポイント取得後: データを書き込み禁止 (read-only) に設定
    \item 書き込み違反(SEGV発生)時: 違反対象ページを記録; 書き込み可に設定
    \item 次回のチェックポイント取得時: 2.で記録されたページのみ保存
    \end{enumerate}
    \begin{center}
      \includegraphics[width=0.8\textwidth]{out/pdf/svg/checkpoint_2.pdf}
    \end{center}
  \end{itemize}
\end{frame}


%%%%%%%%%%%%%%%%% 
\begin{frame}
  \frametitle{並行チェックポインティング}
  \begin{itemize}
  \item 目的: ユーザプログラムの停止時間を短くする
  \item 基本: チェックポイントの保存と, ユーザプログラムの実行を並行して行う
  \item 手法:
    \begin{enumerate}
    \item チェックポイントに到達した際, データを保存するかわりに, データを書き込み禁止に設定
    \item すぐにユーザプログラムを再開, それと並行してチェックポイントを保存
    \item あるデータに書き込みが起きたらそのページの(書き込み前の)データを保存
    \end{enumerate}
  \end{itemize}
  \begin{center}
    \includegraphics[width=0.8\textwidth]{out/pdf/svg/checkpoint_3.pdf}
  \end{center}
\end{frame}

%%%%%%%%%%%%%%%%%
\section{ネットワークページング}
%%%%%%%%%%%%%%%%% 

%%%%%%%%%%%%%%%%% 
\begin{frame}
  \frametitle{ネットワークページング}
  \begin{itemize}
  \item 主記憶の少ない計算機が, ページング領域として, 2次記憶の代わりに,
    主記憶を多く搭載したサーバ(メモリサーバ)の主記憶を使う(主記憶の拡張)

  \item 2次記憶の速度 $<$ ネットワークの速度の場合に有効
  \end{itemize}

    \begin{center}
\includegraphics[width=0.6\textwidth]{out/pdf/svg/network_paging_1.pdf}      
    \end{center}
    

\end{frame}

    
%%%%%%%%%%%%%%%%% 
\begin{frame}
  \frametitle{参考: 2次記憶とネットワークの速度}
  \begin{itemize}
  \item 2次記憶

    {\small
        \begin{tabular}{|l|l|}\hline
          SATA HDD & $\approx$ 100MB/sec \\
          SATA SSD & $\approx$ 500MB/sec \\
          NVMe SSD & $\approx$ 1-5GB/sec \\
          Optane Persistent Memory & $\approx$ 2-7GB/sec \\\hline
        \end{tabular}}

  \item ネットワーク

    {\small
        \begin{tabular}{|l|l|l|}\hline
          Gigabit Ethernet & ラップトップ & $\approx$ 100MB/sec \\
          10Gb Ethernet & 普通のサーバ & $\approx$ 1GB/sec \\
          100Gb Ethernet & 高性能サーバ & $\approx$ 10GB/sec \\
          400Gbps Infiniband NDR 4x & スパコン & $\approx$ 50GB/sec \\\hline
        \end{tabular}}

  \item 注: 速度は構成(ディスク数やネットワークのリンク数)によって変わる
  \end{itemize}
\end{frame}

%%%%%%%%%%%%%%%%% 
\begin{frame}
  \frametitle{ネットワークページング アルゴリズム}
  \begin{itemize}
  \item 前提:
    \begin{itemize}
    \item クライアントが使える物理メモリ量 $= P$ ページ
    \item クライアントが用いるメモリ領域 $= L$ ($> P$) ページ 
    \end{itemize}
  \item クライアントは$\leq P$ページだけを(mmapで)割り当てた状態
  \item 残りの$(L -P)$ページにアクセスすると, SEGVが発生 $\rightarrow$
    \begin{enumerate}
    \item すでに$P$ページ割当済みであれば適当なページをメモリサーバに送信; 解放 (munmap)
    \item アクセされたページを割当て (mmap)
    \end{enumerate}
  \item $\approx$ OSのメモリ管理をユーザプログラム内で行っている
    \begin{itemize}
    \item OS内のページフォルト$\approx$ユーザプログラムのSEGV
    \end{itemize}
  \end{itemize}
\end{frame}

%%%%%%%%%%%%%%%%%
\section{分散共有メモリ}
%%%%%%%%%%%%%%%%% 

%%%%%%%%%%%%%%%%% 
\begin{frame}
  \frametitle{分散共有メモリとは}
  \begin{itemize}
  \item 複数の計算機で「あたかも」共有メモリを持っているかのように見せる技術
  \item $\approx$ OpenMPやスレッドプログラミングを用いた並列処理を, 複数の計算機でも可能にする
    \begin{itemize}
    \item $\Leftrightarrow$ ソケットやメッセージによる通信
    \end{itemize}
  \end{itemize}

  \begin{center}
    \includegraphics[width=0.5\textwidth]{out/pdf/svg/network_paging_2.pdf}      
  \end{center}
    
\end{frame}

%%%%%%%%%%%%%%%%% 
\begin{frame}
  \frametitle{そもそも共有メモリとは}
  \begin{itemize}
  \item 「あるスレッドが書き込みをするだけで他のスレッドにそれが伝わる($\ast$)」ことが本質

    \begin{tabular}{c|cl}
      $T_0$               & $T_1$             & \\\hline
      {\tt a[100] = 200;} &                   & \\
      \ldots              & \ldots            & \\
                          & {\tt x = a[100];} & {\tt /* 200 */} \\
    \end{tabular}

    
  \item \ao{ハードウェア共有メモリ}
    \begin{itemize}
    \item 普通, 共有メモリと言えばこの意味
    \item 1つの計算機の中で複数のCPU (やその中のコア)が物理的に同じ主記憶を使っている
    \end{itemize}
  \item \ao{「分散」共有メモリ}
    \begin{itemize}
    \item 複数の計算機間で, ($\ast$)を実現するソフトウェア
    \end{itemize}
  \item まずはハードウェア共有メモリの実現方法を述べる
  \item 分散共有メモリでも考え方は共通
  \end{itemize}
\end{frame}


%%%%%%%%%%%%%%%%%
\subsection{ハードウェア共有メモリ}
%%%%%%%%%%%%%%%%%

%%%%%%%%%%%%%%%%% 
\begin{frame}
  \frametitle{ハードウェア共有メモリ}
  \begin{itemize}
  \item 1つの計算機の中では同一の物理メモリを使っていると言っても,
    CPUはメモリアクセス命令のたびに主記憶にアクセスしているわけではない(キャッシュ)

    \begin{center}
      \includegraphics[width=0.5\textwidth]{out/pdf/svg/msi_1.pdf}
    \end{center}
    
  \item したがって, 1つの計算機内でも($\ast$)を実現するのは容易ではない
  \item \ldots どころか, 極めて複雑な, \ao{キャッシュ一貫性 (cache coherency) プロトコル}
    で実現されている
  % \item 要点
  %   \begin{itemize}
  %   \item 書き込む時は他のコアに「俺以外は勝手に読んじゃダメ」と通知
  %   \item 読み込む時は他のコアに「誰も勝手に書いちゃダメ」と通知
  %   \end{itemize}
  \end{itemize}
\end{frame}


\newcommand{\Mbox}{{\textcolor{green}{\rule{3mm}{4mm}}}}
\newcommand{\Shbox}{{\textcolor{yellow}{\rule{3mm}{2.5mm}}}}
\newcommand{\Ibox}{{\textcolor{red}{\rule{3mm}{1mm}}}}

\iffalse
%%%%%%%%%%%%%%%%%%%%%%%%%%%%%%%%%% 
\begin{frame}
\frametitle{Cache states and transaction}

\begin{itemize}
\item suppose a processor reads or writes an address and
  finds a line caching it
\item what happens when the line is in each state:
\begin{tabular}{|c|c|c|c|}\hline
      & Modified & Shared     & Invalid   \\\hline
read  & hit      & hit        & \ao{read miss} \\
write & hit      & \ao{write miss} & \ao{read miss; write miss} \\\hline
\end{tabular}

\item<2-> \ao{read miss:} $\rightarrow$ 
  \begin{itemize}
  \item there may be a cache holding it in Modified state \ao{\em (owner)}
  \item searches for the owner and if found, downgrade it to Shared
  \item 
\Ibox , \Mbox , \Ibox , [\Ibox], \Ibox , \ldots
$\Rightarrow$ 
\Ibox , \Shbox , \Ibox , [\Shbox], \Ibox , \ldots
  \end{itemize}
\item<3-> \ao{write miss:} $\rightarrow$
  \begin{itemize}
  \item there may be caches holding it in Shared state \ao{\em (sharer)}
  \item searches for sharers and downgrade them to Invalid
  \item 
\Shbox , \Ibox , \Shbox , [\Shbox], \Ibox , \ldots
$\Rightarrow$ 
\Ibox , \Ibox , \Ibox , [\Mbox], \Ibox , \ldots
  \end{itemize}
\end{itemize}
\end{frame}
\fi

%%%%%%%%%%%%%%%%% 
\begin{frame}
  \frametitle{最も単純なキャッシュ一貫性プロトコル --- Modify-Shared-Invalid (MSI)}
  \begin{itemize}
  \item 各コアは一定量($\ll$ 主記憶量)のキャッシュを持つ
    \begin{center}
      \includegraphics[width=0.5\textwidth]{out/pdf/svg/msi_1.pdf}
    \end{center}
  \item キャッシュ内の各ブロック(管理の単位. 普通は64バイト)は以下の3つのどれかの状態
    \begin{itemize}
    \item Modified (\Mbox) $\iff$ 他のコアに通知せずにwrite/read両方が可能
    \item Shared (\Shbox) $\iff$ 他のコアに通知せずにreadが可能 
    \item Invalid (\Ibox) $\iff$ 他のコアに通知せずには何もできない
    \end{itemize}
  \end{itemize}
\end{frame}

%%%%%%%%%%%%%%%%% 
\begin{frame}
  \frametitle{MSIの動作例}
\only<1>{Modified (\Mbox) が不在の状態}%
\only<2>{誰かが書き込む}%
\only<3>{Shared (\Shbox) の全員をInvalid (\Ibox)に}%
\only<4>{誰かが読み込む}%
\only<5>{Modified (\Mbox) の人をShared (\Shbox)に}%
\only<6>{Shared (\Shbox) は何人いても良い}%

\begin{center}
\only<1>{\includegraphics[width=0.6\textwidth]{out/pdf/svg/msi_2.pdf}}%
\only<2>{\includegraphics[width=0.6\textwidth]{out/pdf/svg/msi_3.pdf}}%
\only<3>{\includegraphics[width=0.6\textwidth]{out/pdf/svg/msi_4.pdf}}%
\only<4>{\includegraphics[width=0.6\textwidth]{out/pdf/svg/msi_5.pdf}}%
\only<5>{\includegraphics[width=0.6\textwidth]{out/pdf/svg/msi_6.pdf}}%
\only<6>{\includegraphics[width=0.6\textwidth]{out/pdf/svg/msi_7.pdf}}%
\end{center}

\end{frame}


%%%%%%%%%%%%%%%%% 
\begin{frame}
  \frametitle{MSIの不変条件と動作}
  \begin{itemize}
  \item 1ブロックに対するキャッシュの状態は以下のどちらかを保つ
    \begin{itemize}
    \item 1 個のModified (\Mbox) $+$ 任意個の Invalids
    \item 任意個の Shared (\Shbox) / Invalid (\Ibox)
    \end{itemize}
  \item 状態とアクションに応じてプロトコルを発動

\begin{tabular}{|c|c|c|c|}\hline
      & Modified & Shared     & Invalid   \\\hline
read  & hit      & hit        & \ao{read miss} \\
write & hit      & \ao{write miss} & \ao{read miss; write miss} \\\hline
\end{tabular}

\begin{itemize}
\item hit : 何もしない(cache hit)
\item \ao{write miss} : 他のコアをInvalid (\Ibox)に
\item \ao{read miss} : ModifiedなコアをShared (\Shbox)に
\end{itemize}
    
  \end{itemize}
\end{frame}


%%%%%%%%%%%%%%%%%
\subsection{分散共有メモリ}
%%%%%%%%%%%%%%%%%

%%%%%%%%%%%%%%%%% 
\begin{frame}
  \frametitle{分散共有メモリの動作}
  \begin{itemize}
  \item $\approx$ MSIをソフトウェアで実現
  \item 管理の単位はページ(当然)で, キャッシュの状態を, ページの保護属性(mprotect)で実現
    \begin{itemize}
    \item M : 読み書き許可 ({\tt PROTO\_READ|PROTO\_WRITE})
    \item S : 読み出しのみ許可 ({\tt PROTO\_READ})
    \item I : 読み書き禁止 ({\tt PROTO\_NONE})
    \end{itemize}
  \end{itemize}


  \begin{center}
    \only<1>{\includegraphics[width=0.5\textwidth]{out/pdf/svg/network_paging_3.pdf}}%
    \only<2>{\includegraphics[width=0.5\textwidth]{out/pdf/svg/network_paging_4.pdf}}%
    \only<3>{\includegraphics[width=0.5\textwidth]{out/pdf/svg/network_paging_5.pdf}}%
    \only<4>{\includegraphics[width=0.5\textwidth]{out/pdf/svg/network_paging_6.pdf}}%
    \only<5>{\includegraphics[width=0.5\textwidth]{out/pdf/svg/network_paging_7.pdf}}%
    \only<6>{\includegraphics[width=0.5\textwidth]{out/pdf/svg/network_paging_8.pdf}}%
    \only<7>{\includegraphics[width=0.5\textwidth]{out/pdf/svg/network_paging_9.pdf}}%
    \only<8>{\includegraphics[width=0.5\textwidth]{out/pdf/svg/network_paging_10.pdf}}%
    \only<9>{\includegraphics[width=0.5\textwidth]{out/pdf/svg/network_paging_11.pdf}}%
  \end{center}
\end{frame}


%%%%%%%%%%%%%%%%% 
\section{インクリメンタル Garbage Collection}
%%%%%%%%%%%%%%%%% 

%%%%%%%%%%%%%%%%% 
\begin{frame}
  \frametitle{Garbage Collection (GC), インクリメンタル GC}

  \begin{itemize}
  \item \ao{Garbage Collection (GC)}
    \begin{itemize}
    \item 今後使われないメモリ領域を自動的に発見・回収する技術
    \item C言語で言うならば, mallocした領域をfreeしなくてもよくする技術
    \item どうやって? $\Rightarrow$ 次スライド
    \end{itemize}

  \item \ao{インクリメンタルGC}
    \begin{itemize}
    \item GCを「少しずつ」行うことで,
      プログラムの「停止時間」を短くする技術
    \end{itemize}

  \end{itemize}
\end{frame}

%%%%%%%%%%%%%%%%% 
\begin{frame}
\frametitle{GCの基本原理}
\begin{itemize}
\item プログラムが使うメモリ領域は, プログラムの大域変数/局所変数(\ao{ルート}),
  およびそこからポインタを任意回たどってたどり着く領域(だけ)である
\item GCの動作
  \begin{enumerate}
  \item ルートからポインタをたどって, 使われうるメモリ領域を見つけていく
  \item これ以上たどれるポインタがなくなったところで,
    見つけられていない領域を回収
  \end{enumerate}
\item {\scriptsize 注: C言語のようにポインタを使って任意のアドレスにアクセスできる言語では,
    厳密に正しいとは言えないが, 仕様上, そのようなプログラムの動作は未定義で,
    「そのようなプログラムはサポート外」としても有用性は損なわれないだろう}
\end{itemize}

\begin{center}
\includegraphics[width=0.4\textwidth]{out/pdf/svg/ms_1.pdf}
\end{center}

\end{frame}

%%%%%%%%%%%%%%%%% 
\begin{frame}
\frametitle{GCの動作 $\approx$ グラフのノード探索}
\begin{itemize}
\item ノード : メモリを割り当てた単位(Cであればmalloc 1回で割り当てた領域)
\item 辺 : ポインタ
\item GC $\approx$ ルートから\ao{到達可能}なノードを列挙し, それ以外を回収
\end{itemize}

\begin{center}
\only<1>{\includegraphics[width=0.7\textwidth]{out/pdf/svg/ms_1.pdf}}%
\only<2>{\includegraphics[width=0.7\textwidth]{out/pdf/svg/ms_2.pdf}}%
\only<3>{\includegraphics[width=0.7\textwidth]{out/pdf/svg/ms_3.pdf}}%
\only<4>{\includegraphics[width=0.7\textwidth]{out/pdf/svg/ms_4.pdf}}%
\only<5>{\includegraphics[width=0.7\textwidth]{out/pdf/svg/ms_5.pdf}}%
\only<6>{\includegraphics[width=0.7\textwidth]{out/pdf/svg/ms_6.pdf}}
\only<7>{\includegraphics[width=0.7\textwidth]{out/pdf/svg/ms_7.pdf}}
\end{center}
\end{frame}

%%%%%%%%%%%%%%%%% 
\begin{frame}
\frametitle{GCとプログラミング言語}
\begin{itemize}
\item GCはプログラミング言語に不可欠の機能
    \begin{itemize}
    \item 詳しくは4年生の授業「プログラミング言語」にて
    \item インクリメンタル GCのために仮想記憶APIが役に立つのでここで取り上げます
    \end{itemize}
  
\item C/C++/Fortran以外のほとんどのプログラミング言語(Python, Ruby, Julia, etc.)はGCを持つ
  \begin{itemize}
  \item おかげでプログラマは自分でメモリを回収する(free/deleteなどを呼ぶ)必要がない(安全)
  \end{itemize}

\item C/C++ からライブラリとして使えるGCがある
  \ao{(\href{https://hboehm.info/gc/}{Boehm GC})}
\item 本講義の残りでもこれを使って実験
\end{itemize}
\end{frame}

%%%%%%%%%%%%%%%%% 
\begin{frame}
\frametitle{Boehm GC}
\begin{itemize}
\item {\tt malloc}の代わりに{\tt GC\_MALLOC}という関数を呼ぶ(だけ)
  \begin{itemize}
  \item このとき, 必要なタイミングで($\approx$ OSから取得した領域を使い果たしたら)
    GCが起動され, メモリを回収
  \item OSからどれだけメモリを取得するかは調節可能(詳細省略)
  \end{itemize}
  
\item {\tt free}は呼ばなくて良い(使われない領域を勝手に回収・再利用してくれる)
\end{itemize}
\end{frame}

%%%%%%%%%%%%%%%%% 
\begin{frame}
\frametitle{普通の (インクリメンタルでない) GCの問題点}
\begin{itemize}
\item \mura{到達可能なノードを全て発見し終えるまで, 一切メモリが回収できない}
  \begin{itemize}
  \item 例えば 10 GBのデータを使っているプログラムでGCがおきると,
    メモリを10 GB分触り終えるまで, (1バイトも)メモリが回収できない
  \end{itemize}
\item \mura{その間, ユーザプログラムは動けない}
  \begin{itemize}
  \item GCがグラフをたどっている間にグラフが変化すると,
    面倒なことになりそう
  \end{itemize}
\item $\Rightarrow$
  プログラムが時折経験する「停止時間」が(プログラムが使っているデータ量に応じて)長くなる
\end{itemize}
\begin{center}
\includegraphics[width=0.8\textwidth]{out/pdf/svg/incremental_basic_1.pdf}
\end{center}
\end{frame}


%%%%%%%%%%%%%%%%% 
\begin{frame}
\frametitle{インクリメンタルGC}
\begin{itemize}
\item GCの「停止時間 ($\neq$ 総時間)」を短くするGC
  \begin{itemize}
  \item 実時間性や応答性を必要とするアプリケーション
  \end{itemize}
\item 停止時間 $\approx$ 到達可能なノード全てを走査する時間
\item インクリメンタルGCはその走査を,
  少しずつ「細切れ」に行って,停止時間を短くする
  \begin{itemize}
  \item e.g., 10GB「一気に」走査せずに,10MBずつ1000回に分けて走査
  \end{itemize}
\begin{center}
\includegraphics[width=0.8\textwidth]{out/pdf/svg/incremental_basic_2.pdf}
\end{center}
\end{itemize}
\end{frame}

%%%%%%%%%%%%%%%%% 
\begin{frame}
\frametitle{インクリメンタルGCの困難さと解決策}
\begin{itemize}
\item グラフを走査している間にユーザプログラムが動作
\item $\Rightarrow$ 走査中にグラフが変化する(書き換わる)
\item<4-> いくつか方法があるが, Boehm GCでは,
  書き換わったノード(ページ)から, 後でもう一度たどり直す
  \begin{itemize}
  \item これで解決する事をきちんと説明すると長くなるので省略 (宣伝: 来学期「プログラミング言語」)
  \end{itemize}
\item<5-> 書き換わったページを検出するのに, 仮想記憶API (mprotect)を使う
\end{itemize}

\begin{center}
\only<1>{\includegraphics[width=0.6\textwidth]{out/pdf/svg/ms_8.pdf}}%
\only<2>{\includegraphics[width=0.6\textwidth]{out/pdf/svg/ms_9.pdf}}%
\only<3->{\includegraphics[width=0.6\textwidth]{out/pdf/svg/ms_10.pdf}}%
\end{center}
\end{frame}

%%%%%%%%%%%%%%%%% 
\begin{frame}
\frametitle{Boehm GCのインクリメンタルGC}
\begin{enumerate}
\item {\tt GC\_MALLOC}が呼ばれた際, あるタイミングでGCを開始; 全領域を「書き込み不可({\tt PROT\_READ})」に設定
\item その後{\tt GC\_MALLOC}が呼ばれた際, 到達可能な領域を「少しずつ」辿る(合間にユーザプログラムも動く)
\item ユーザが書き込み不可のページに書き込みを行ったら, そのページを記録しておく
\item グラフをたどり終えたら改めて, 書き込みまれたページから辿れるオブジェクトを辿る
\end{enumerate}

\begin{center}
\includegraphics[width=0.8\textwidth]{out/pdf/svg/incremental_basic_3.pdf}
\end{center}
\end{frame}

  
\end{document}
