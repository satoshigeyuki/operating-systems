\documentclass[12pt,dvipdfmx]{beamer}
\institute{}
\author{田浦健次朗}
\date{}

\usepackage{graphicx}
\DeclareGraphicsExtensions{.pdf}
\DeclareGraphicsExtensions{.eps}
\graphicspath{{out/}{out/tex/}{out/tex/gpl/}{out/tex/svg/}{out/tex/lsvg/}{out/tex/dot/}}
% \graphicspath{{out/}{out/tex/}{out/pdf/}{out/eps/}{out/tex/gpl/}{out/tex/svg/}{out/pdf/dot/}{out/pdf/gpl/}{out/pdf/img/}{out/pdf/odg/}{out/pdf/svg/}{out/eps/dot/}{out/eps/gpl/}{out/eps/img/}{out/eps/odg/}{out/eps/svg/}}
\usepackage{listings,jlisting}
\usepackage{fancybox}
\usepackage{hyperref}
\usepackage{color}

%%%%%%%%%%%%%%%%%%%%%%%%%%%
%%% themes
%%%%%%%%%%%%%%%%%%%%%%%%%%%
\usetheme{default} % Szeged
%% no navigation bar
% default boxes Bergen Boadilla Madrid Pittsburgh Rochester
%% tree-like navigation bar
% Antibes JuanLesPins Montpellier
%% toc sidebar
% Berkeley PaloAlto Goettingen Marburg Hannover Berlin Ilmenau Dresden Darmstadt Frankfurt Singapore Szeged
%% Section and Subsection Tables
% Copenhagen Luebeck Malmoe Warsaw

%%%%%%%%%%%%%%%%%%%%%%%%%%%
%%% innerthemes
%%%%%%%%%%%%%%%%%%%%%%%%%%%
% \useinnertheme{circles}	% default circles rectangles rounded inmargin

%%%%%%%%%%%%%%%%%%%%%%%%%%%
%%% outerthemes
%%%%%%%%%%%%%%%%%%%%%%%%%%%
% outertheme
% \useoutertheme{default}	% default infolines miniframes smoothbars sidebar sprit shadow tree smoothtree


%%%%%%%%%%%%%%%%%%%%%%%%%%%
%%% colorthemes
%%%%%%%%%%%%%%%%%%%%%%%%%%%
\usecolortheme{seahorse}
%% special purpose
% default structure sidebartab 
%% complete 
% albatross beetle crane dove fly seagull 
%% inner
% lily orchid rose
%% outer
% whale seahorse dolphin

%%%%%%%%%%%%%%%%%%%%%%%%%%%
%%% fontthemes
%%%%%%%%%%%%%%%%%%%%%%%%%%%
\usefonttheme{serif}  
% default professionalfonts serif structurebold structureitalicserif structuresmallcapsserif

%%%%%%%%%%%%%%%%%%%%%%%%%%%
%%% generally useful beamer settings
%%%%%%%%%%%%%%%%%%%%%%%%%%%
% 
\AtBeginDvi{\special{pdf:tounicode EUC-UCS2}}
% do not show navigation
\setbeamertemplate{navigation symbols}{}
% show page numbers
\setbeamertemplate{footline}[frame number]

%%%%%%%%%%%%%%%%%%%%%%%%%%%
%%% define some colors for convenience
%%%%%%%%%%%%%%%%%%%%%%%%%%%

\newcommand{\mido}[1]{{\color{green}#1}}
\newcommand{\mura}[1]{{\color{purple}#1}}
\newcommand{\ore}[1]{{\color{orange}#1}}
\newcommand{\ao}[1]{{\color{blue}#1}}
\newcommand{\aka}[1]{{\color{red}#1}}

\setbeamercolor{ex}{bg=cyan!20!white}

\definecolor{UniBlue}{RGB}{20,20,250} 
\setbeamercolor{structure}{fg=UniBlue} % 見出しカラー

\iffalse
%%%%%%%%%%%%%%%%%%%%%%%%%%%
%% customize beamer template
%% https://www.opt.mist.i.u-tokyo.ac.jp/~tasuku/beamer.html
%%%%%%%%%%%%%%%%%%%%%%%%%%%

%\renewcommand{\familydefault}{\sfdefault}  % 英文をサンセリフ体に
%\renewcommand{\kanjifamilydefault}{\gtdefault}  % 日本語をゴシック体に
\usefonttheme{structurebold} % タイトル部を太字
\setbeamerfont{alerted text}{series=\bfseries} % Alertを太字
\setbeamerfont{section in toc}{series=\mdseries} % 目次は太字にしない
\setbeamerfont{frametitle}{size=\Large} % フレームタイトル文字サイズ
\setbeamerfont{title}{size=\LARGE} % タイトル文字サイズ
\setbeamerfont{date}{size=\small}  % 日付文字サイズ

\definecolor{UniBlue}{RGB}{0,150,200} 
\definecolor{AlertOrange}{RGB}{255,76,0}
\definecolor{AlmostBlack}{RGB}{38,38,38}
\setbeamercolor{normal text}{fg=AlmostBlack}  % 本文カラー
\setbeamercolor{structure}{fg=UniBlue} % 見出しカラー
\setbeamercolor{block title}{fg=UniBlue!50!black} % ブロック部分タイトルカラー
\setbeamercolor{alerted text}{fg=AlertOrange} % \alert 文字カラー
\mode<beamer>{
    \definecolor{BackGroundGray}{RGB}{254,254,254}
    \setbeamercolor{background canvas}{bg=BackGroundGray} % スライドモードのみ背景をわずかにグレーにする
}


%フラットデザイン化
\setbeamertemplate{blocks}[rounded] % Blockの影を消す
\useinnertheme{circles} % 箇条書きをシンプルに
\setbeamertemplate{navigation symbols}{} % ナビゲーションシンボルを消す
\setbeamertemplate{footline}[frame number] % フッターはスライド番号のみ

%タイトルページ
\setbeamertemplate{title page}{%
    \vspace{2.5em}
    {\usebeamerfont{title} \usebeamercolor[fg]{title} \inserttitle \par}
    {\usebeamerfont{subtitle}\usebeamercolor[fg]{subtitle}\insertsubtitle \par}
    \vspace{1.5em}
    \begin{flushright}
        \usebeamerfont{author}\insertauthor\par
        \usebeamerfont{institute}\insertinstitute \par
        \vspace{3em}
        \usebeamerfont{date}\insertdate\par
        \usebeamercolor[fg]{titlegraphic}\inserttitlegraphic
    \end{flushright}
}
\fi

%%%%%%%%%%%%%%%%%%%%%%%%%%%
%%% how to typset code
%%%%%%%%%%%%%%%%%%%%%%%%%%%

\lstset{language = C,
numbers = left,
numberstyle = {\tiny \emph},
numbersep = 10pt,
breaklines = true,
breakindent = 40pt,
frame = tlRB,
frameround = ffft,
framesep = 3pt,
rulesep = 1pt,
rulecolor = {\color{blue}},
rulesepcolor = {\color{blue}},
flexiblecolumns = true,
keepspaces = true,
basicstyle = \ttfamily\scriptsize,
identifierstyle = ,
commentstyle = ,
stringstyle = ,
showstringspaces = false,
tabsize = 4,
escapechar=\@,
}

\AtBeginSection[]
{
\begin{frame}
\frametitle{}
\tableofcontents[currentsection]
\end{frame}
}

\AtBeginSubsection[]
{
\begin{frame}
\frametitle{}
\tableofcontents[currentsection,currentsubsection]
\end{frame}
}


\title{メモリ管理(仮想記憶)}
\begin{document}
\maketitle

%%%%%%%%%%%%%%%%%%%%%%%%%%%%%%%%%% 
\begin{frame}
\frametitle{目次}
\tableofcontents
\end{frame}

%%%%%%%%%%%%%%%%% 
\section{メモリの復習}
%%%%%%%%%%%%%%%%% 

%%%%%%%%%%%%%%%%% 
\begin{frame}
  \frametitle{メモリ}
  \begin{itemize}
  \item コンピュータには\ao{主記憶}(\ao{メインメモリ},
    しばしば単に\ao{メモリ})
    が搭載されている
    \begin{itemize}
    \item 典型的な大きさ: 4GB (ノートPC) 〜 256 GB (サーバ)
    \item 1GB $= 2^{30}$B $= 1073741824$B
    \item 1B (1バイト) $=$ 8 bit
    \end{itemize}

  \item 1B ごとに\ao{\bf 番地(アドレス)}という通し番号がついている
    \begin{itemize}
    \item 例: 8GBのメモリ 0 〜 $2^{33} - 1 = 8589934591$
    \end{itemize}

  \item CPUは\ao{アドレスを指定してメモリを読み書き}する(load命令, store命令)
  \end{itemize}

  \begin{center}
  \includegraphics[width=0.5\textwidth]{out/pdf/svg/memory.pdf}
  \end{center}
  
\end{frame}

%%%%%%%%%%%%%%%%% 
\begin{frame}[fragile]
  \frametitle{プログラム言語とメモリ}
  \begin{itemize}
  \item プログラムが使うデータはほぼ全てメモリの中にある
    \begin{itemize}
    \item 編集中の文書, スプレッドシート, \ldots
    \item 開いているウェブページ, \ldots
    \end{itemize}

  \item プログラム内では「変数」「配列」などに格納される

  % \item コンピュータがやっていることは大雑把に言えば,
  %   \begin{itemize}
  %   \item 外部装置(HDD, ネットワークなど)からメモリにデータを移動
  %   \item メモリからデータ(数字)を読む
  %   \item CPUが計算して別のデータを作り出す
  %   \item メモリにデータを書き込む
  %   \item メモリから外部装置(HDD, ネットワークなど)にデータを移動
  %   \end{itemize}
  \end{itemize}
  \begin{center}
    \includegraphics[width=0.7\textwidth]{out/pdf/svg/memory_apps.pdf}
  \end{center}
\end{frame}
    
%%%%%%%%%%%%%%%%% 
\begin{frame}
  \frametitle{プログラム言語とメモリ}
  \begin{itemize}
  \item C/C++でメモリを使う手段
    \begin{itemize}
    \item 大域変数・配列
    \item 局所変数・配列
    \item ヒープ
    \end{itemize}
  \end{itemize}

  \begin{center}
    \includegraphics[width=1.0\textwidth]{out/pdf/svg/memory_program.pdf}
  \end{center}
\end{frame}

%%%%%%%%%%%%%%%%% 
\begin{frame}[fragile]
  \frametitle{プログラムとアドレス}
  \begin{itemize}
  \item 変数・配列$\approx$アドレス(実際は単なる整数)に名前を付けたもの
  \item C言語のポインタ$=$アドレス
    ($\leftarrow$ ポインタがよくわからないという人へ)
  \item ポインタ変数はアドレスを格納する変数
  \item C言語では「変数」「配列」に対応するアドレスをあからさまに
    (整数として)見ることが可能
\begin{lstlisting}
printf("%ld %ld %ld %ld\n", &x, a, &y, p);
\end{lstlisting}
    
  \end{itemize}
\end{frame}

%%%%%%%%%%%%%%%%% 
\section{論理アドレス空間}
%%%%%%%%%%%%%%%%% 

%%%%%%%%%%%%%%%%% 
\begin{frame}
  \frametitle{論理アドレスと物理アドレス}
  \begin{itemize}
  \item プロセスごとに\ao{\bf 論理(仮想)アドレス空間}
  \item プロセスが扱うアドレスはそのプロセスの
    論理アドレス空間内のアドレス:
    \ao{\bf 論理(仮想)アドレス}
  \item 物理メモリを実際にアクセスするときのアドレス: \ao{\bf 物理アドレス}
  \end{itemize}
  \begin{center}
    \includegraphics[width=1.0\textwidth]{out/pdf/svg/memory_many_programs.pdf}
  \end{center}
\end{frame}

%%%%%%%%%%%%%%%%% 
\begin{frame}
  \frametitle{論理アドレス空間}
  \begin{enumerate}
  \item \ao{1プロセスに1論理アドレス空間}
  \item アドレスの範囲は物理メモリの量によらない
    \begin{itemize}
    \item ソフト(OS)の設計で決まる
    \item 例: Linux x86-64 (Intelの64 bit CPU)の場合: 0 〜 $2^{48} - 1$
    \item 注: 同じ論理アドレスが全プロセスにある
    \item プロセス$P$の10000番地 $\neq$ プロセス$Q$の10000番地
      (対応する物理アドレスが異なる)
    \end{itemize}
  \item 論理アドレスから物理アドレスへの変換\ao{\bf (アドレス変換)}
    をCPUが行う
  \item 対応する物理アドレスが\ao{ない場合もある}
  \end{enumerate}
\end{frame}

%%%%%%%%%%%%%%%%% 
\section{メモリ管理ユニット}
%%%%%%%%%%%%%%%%% 

%%%%%%%%%%%%%%%%% 
\begin{frame}
  \frametitle{メモリ管理ユニット}
  \begin{itemize}
  \item Memory Management Unit (MMU)
  \item CPU内のハードウェアで\ao{全メモリアクセスに介在}
  \item (論理アドレス空間, 論理アドレス)に対し以下を行う
    \begin{enumerate}
    \item アクセス権(読み・書き・実行権限)の検査
    \item 対応する物理アドレスが存在するかの検査
    \item 存在していればそのアドレスをアクセス
    \end{enumerate}
  \end{itemize}
  \begin{center}
    \includegraphics[width=1.0\textwidth]{out/pdf/svg/mmu_check.pdf}
  \end{center}
\end{frame}

%%%%%%%%%%%%%%%%% 
\begin{frame}
  \frametitle{MMUによって可能になること}
  \begin{enumerate}
  \item \ao{プロセス間でメモリの分離(保護)}
    \begin{itemize}
    \item あるプロセスが他のプロセスのデータを見たり壊したり,
      決してできないようになる
    \end{itemize}
  \item \ao{カーネル(OS)データの保護}
  \item \ao{物理メモリ量を越えたメモリ割り当て}
  \item \ao{\bf 要求時ページング(on demand paging):}
    物理メモリを確保せずに, メモリ割り当てを(高速に)行える
  \end{enumerate}
  \begin{center}
    \includegraphics[width=1.0\textwidth]{out/pdf/svg/memory_many_programs.pdf}
  \end{center}
\end{frame}

%%%%%%%%%%%%%%%%% 
\section{アドレス変換の実際}
%%%%%%%%%%%%%%%%% 

%%%%%%%%%%%%%%%%% 
\begin{frame}
  \frametitle{アドレス変換の要件}
  \begin{itemize}
  \item 実現すべきもの $=$ 論理アドレス空間ごとに一つの

    \begin{center}
      \ao{論理アドレス $\rightarrow$
        (アクセス許可, 物理アドレス) の写像}
    \end{center}
    
  \item 例: Intel 64 bit の論理アドレス48 bit
    
    \begin{center}
      \includegraphics[width=1.0\textwidth]{out/pdf/svg/addr_page_1.pdf}
    \end{center}
    
  \item 文字通りすべてのアドレス(1バイトごと)に,
    対応する情報(物理アドレスのサイズ $+$ 数bit
    $\approx$ 8バイト)を持たせるのは非現実的
  \item $2^{48} \times 8 \times \mbox{プロセス数}
    = 2\mbox{PB} \times \mbox{プロセス数}$
  \item $\Rightarrow$ \ao{ページ}
  \end{itemize}
\end{frame}

%%%%%%%%%%%%%%%%% 
\begin{frame}
  \frametitle{ページ}
  \begin{itemize}
  \item \ao{\bf ページ} : 最下位$A$ bit以外が共通の, $2^A$個のアドレス
    (から成る領域)
    \begin{itemize}
    \item その$A$ bitを\ao{\bf (ページ内)オフセット}と呼ぶ
    \item それ以外の部分 : \ao{\bf 論理ページ番号}と呼ぶ
    \item ページは連続する$2^{A}$バイト
    \end{itemize}
  \item 典型: $A = 12$ (4KB) または13 (8KB)
  \end{itemize}

  \begin{center}
    \includegraphics[width=\textwidth]{out/pdf/svg/addr_page_3.pdf}
  \end{center}
\end{frame}

%%%%%%%%%%%%%%%%% 
\begin{frame}
  \frametitle{ページ}
  \begin{itemize}
  \item アクセス許可情報, 物理アドレス(のオフセット以外の部分;
    \ao{\bf 物理ページ番号}または\ao{\bf ページフレーム番号})は,
    \ao{ページごとにひとつ}だけ持たせる. i.e.,
    \begin{itemize}
    \item 1ページ内の全アドレスのアクセス許可は共通
    \item 1ページ内の全アドレスは連続したアドレスに変換される
      (オフセット部分は変換されない)
    \end{itemize}
    
  \item 必要なものは, アドレス空間に一つの,
    \begin{center}
    \ao{論理ページ番号 $\rightarrow$ (アクセス許可, 物理ページ番号)}
    \end{center}
    の写像になった
  \end{itemize}
  \begin{center}
    \includegraphics[width=0.8\textwidth]{out/pdf/svg/addr_page_4.pdf}
  \end{center}
\end{frame}

%%%%%%%%%%%%%%%%% 
\begin{frame}
  \frametitle{ページテーブル}
  \begin{itemize}
  \item この写像

    \begin{center}
      \ao{論理ページ番号 $\rightarrow$ (アクセス許可, 物理ページ番号)}
    \end{center}

    を実現するデータ構造を一般に, \ao{\bf ページテーブル}と呼ぶ
      
  \item 実際の構造は?
  \end{itemize}
  
  \begin{columns}
    \begin{column}{0.6\textwidth}
      \begin{itemize}
      \item   論理ページ番号をindexとするただの配列(フラットな配列)だとすると,
        $2^{36} = 64\mbox{G}$要素; 1要素(物理ページ番号$+$アクセス許可) 8バイト
        $\Rightarrow$ 64 G $\times$ 8B $=$ 512GB (依然として非現実的)
      \end{itemize}
    \end{column}
    \begin{column}{0.5\textwidth}
      \begin{center}
        \includegraphics[width=\textwidth]{out/pdf/svg/addr_page_5.pdf}
      \end{center}
    \end{column}
  \end{columns}

  $\Rightarrow$ \ao{\bf 多段ページテーブル}
  
\end{frame}

%%%%%%%%%%%%%%%%% 
\begin{frame}
  \frametitle{多段ページテーブル}
  \begin{itemize}
  \item 論理ページ番号を複数の, 小さいindexに分割
    (e.g., 36 bit $=$ 9 bit $\times$ 4)
  \item ページテーブルは深さ$\leq 4$の木構造
    
%  \item あるindexで表(木構造)をアクセスして次の表を得る
%  {\tt root[$l[39:47]$].next[$l[30:38]$].next[$l[21:29]$].next[$l[12:20]$]}
    
    \begin{center}
      \only<1>{\includegraphics[width=0.7\textwidth]{out/pdf/svg/addr_page_6.pdf}}%
      \only<2>{\includegraphics[width=0.7\textwidth]{out/pdf/svg/addr_page_7.pdf}}%
      \only<3>{\includegraphics[width=0.7\textwidth]{out/pdf/svg/addr_page_8.pdf}}%
      \only<4>{\includegraphics[width=0.7\textwidth]{out/pdf/svg/addr_page_9.pdf}}%
      \only<5->{\includegraphics[width=0.7\textwidth]{out/pdf/svg/addr_page_10.pdf}}%
    \end{center}
  \end{itemize}
\end{frame}

%%%%%%%%%%%%%%%%% 
\begin{frame}
  \frametitle{多段のページテーブルに必要なメモリサイズ}
  \begin{itemize}
  \item 仮定: 36 bit 論理ページ番号 $\rightarrow$ 9 bit $\times$ 4 
  \item 一つの表 : $2^9$ 要素
  \item 必要な表の数 : $1 + 2^9 + 2^{18} + 2^{27} \approx 2^{27}$
  \item メモリサイズ : $8 \mbox{バイト/要素} \times 2^9 \mbox{要素} \times 2^{27} = 512\mbox{GB}$ (あれ?)
  \item 結局最悪の場合, 1ページにつき1つの物理ページが必要と考えれば当然\ldots
  \end{itemize}
\end{frame}

%%%%%%%%%%%%%%%%% 
\begin{frame}
  \frametitle{多段のページテーブルに必要なメモリサイズ}
  \begin{itemize}
  \item ほとんどのプロセスは, 論理アドレス空間のごく一部しか,
    使って(割り当てて)いない
  \item そのような場合のメモリサイズが問題
  \item ただの配列$\rightarrow$依然として最悪の場合と同じだけ必要
  \item 多段ページテーブル$\rightarrow$ほとんどのアドレスの, 下位の表は不要
  \end{itemize}
\end{frame}

%%%%%%%%%%%%%%%%% 
\begin{frame}
  \frametitle{ページごとの情報(ページテーブルエントリ)}
  \begin{itemize}
  \item P : 物理ページが割り当てられているか?
  \item いる($P = 1$)場合:
    \begin{itemize}
    \item W : 書き込み可能ならば1
    \item U : ユーザモードでアクセス可ならば1
    \item A : アクセスされたときに1がsetされる
    \item D : 書き込まれたときに1がsetされる
    \item PFN : 物理ページ番号(ページフレーム番号)
    \end{itemize}
  \end{itemize}
  
  \begin{center}
  \includegraphics[width=1.0\textwidth]{out/pdf/svg/addr_page_2.pdf}
  \end{center}
\end{frame}

%%%%%%%%%%%%%%%%% 
\begin{frame}
  \frametitle{Translation Lookaside Buffer (TLB)}
  \begin{itemize}
  \item 1つの論理アドレスをアクセスする準備---アドレス変換---のために,
    4回のメモリアクセスが必要(避けたい)
    $\Rightarrow$ \ao{\bf Translation Lookaside Buffer (TLB)}
  \item
    (論理アドレス空間, 論理ページ番号) $\rightarrow$ (アクセス許可, 物理ページ番号)
    の写像の一部を保持するプロセッサ内のキャッシュ
  \item $\approx$ 1024要素程度
  \end{itemize}
\end{frame}

%%%%%%%%%%%%%%%%% 
\begin{frame}
  \frametitle{その他 細かめの話}
  \begin{itemize}
  \item \ao{\bf Large pages, huge pages}
    \begin{itemize}
    \item ページ内のオフセットを何bitとするか($=$ どこまでを論理ページ番号(アドレス変換のキー)とするか)で1ページのサイズが変わる
    \item 多段のページテーブルでは, 何段目までを論理ページ番号に使うかで,
      異なるページサイズを同居可能
      \begin{itemize}
      \item 36 + 12 $\Rightarrow$ (通常の) 4KB ページ
      \item 27 + 21 $\Rightarrow$ 2MB ページ
      \item 18 + 30 $\Rightarrow$ 1GB ページ
      \end{itemize}
    \end{itemize}
  \item \ao{57 bit 論理アドレス (5段のページテーブル)}
    Intelの最新(Ice Lake)のアーキテクチャで導入
    \begin{itemize}
    \item 48 bit $\Rightarrow$ 256 TB 論理アドレス空間
    \item 57 bit $\Rightarrow$ 128 PB 論理アドレス空間
    \end{itemize}
  \end{itemize}
\end{frame}

%%%%%%%%%%%%%%%%% 
\section{OSのメモリ割り当てAPI}
%%%%%%%%%%%%%%%%% 

%%%%%%%%%%%%%%%%% 
\begin{frame}
  \frametitle{Unixメモリ割り当てAPI (1)}
  \begin{itemize}
  \item \ao{\tt int $e$ = brk($l$);}
    \begin{itemize}
    \item データ領域の終わりのアドレスを$l$にする
    \item $\Rightarrow$ $x < l$のアドレス$x$が「利用可能」になる(つまりメモリが割り当てられる)
    \end{itemize}
  \item \ao{\tt void * $a$ = sbrk({\it sz});}
    \begin{itemize}
    \item データ領域の終わりのアドレスを{\it sz}バイト増やし,
      増やす前のアドレスを返す
    \item $\Rightarrow$ $x \in [a, a+\mbox{\it sz})$
      のアドレスが利用可能になる(つまりメモリが割り当てられる)
    \item {\tt sbrk(0)}は「データセグメントの終わりのアドレス」を返す
    \end{itemize}
  \end{itemize}
\end{frame}

%%%%%%%%%%%%%%%%% 
\begin{frame}
  \frametitle{余談: なぜ変な名前(brk, sbrk)?}
  \begin{columns}
    \begin{column}{0.6\textwidth}
      \begin{itemize}
      \item 古のUnixでのアドレス空間: 割当て領域を少数の連続領域(セグメント)に限定
        \begin{itemize}
        \item コード(テキスト)セグメント
        \item スタックセグメント
        \item データセグメント
        \end{itemize}
      \item データセグメントの終わりのアドレスを\ao{\bf break 値}と呼んでいた
      \item 現代のOS:
        MMUによりページ単位で任意に割当て・非割当領域を管理可能
      \end{itemize}
    \end{column}
    \begin{column}{0.4\textwidth}
      \includegraphics[width=1.0\textwidth]{out/pdf/svg/segments.pdf}
    \end{column}
  \end{columns}

\end{frame}

%%%%%%%%%%%%%%%%% 
\begin{frame}
  \frametitle{Unixメモリ割り当てAPI (2)}
  \begin{itemize}
  \item \ao{\tt mmap, mremap, munmap} : より柔軟なメモリ割り当て, 開放, ファイルのマッピング(後述)
  \item \ao{\tt int $e$ = mprotect($a$, {\it sz}, {\it prot});}
    \begin{itemize}
    \item メモリ領域のアクセス許可を設定(書き込み, 読み込み不可に)できる
    \end{itemize}
  \end{itemize}
\end{frame}

%%%%%%%%%%%%%%%%% 
\begin{frame}
  \frametitle{メモリ割り当て$=$「仮想」メモリの割当 $\neq$ 「物理」メモリの割当}
  \begin{itemize}
  \item brk, sbrk, mmapなどでメモリを割り当てることの効果
    \begin{itemize}
    \item $=$ 所定の論理アドレスの範囲が\ao{「アクセス可能になる」}
    \item $=$ 「アクセス時にSegmentation Faultがおきなくなる」
    \end{itemize}
  \item 実際の動作:
    \begin{enumerate}
    \item OSが管理するデータ
      (\ao{\bf アドレス空間記述表}, Linux {\tt mm\_struct})に,
      割当て済みであることを記述する(だけ)
    \item 物理メモリはすぐには割り当てられない
    \end{enumerate}
  \end{itemize}
\end{frame}

%%%%%%%%%%%%%%%%% 
\begin{frame}
  \frametitle{メモリ割り当て$=$「仮想」メモリの割当 $\neq$ 「物理」メモリの割当}
  \begin{itemize}
  \item 実際にアクセスが起きた際に
    \begin{enumerate}
    \item CPUがページフォルト例外を発生
    \item OSの例外処理(ページフォルト処理)が発動
      \begin{itemize}
      \item Unix: \ao{\bf マイナー}フォルトと呼び後述のメジャーフォルトと区別
      \end{itemize}
    \item OSがアドレス空間記述表を見て,
      そのアクセスが実際には合法であることが確認されると,
      物理メモリが割り当てられる
      (\ao{\bf 要求時ページング})
    \end{enumerate}
  \end{itemize}
\end{frame}

%%%%%%%%%%%%%%%%% 
\begin{frame}
  \frametitle{メモリ管理動作図(割当て〜要求時ページング)}
  \begin{center}
    \only<1>{\includegraphics[width=1.0\textwidth]{out/pdf/svg/mem_management_1.pdf}}%
    \only<2>{\includegraphics[width=1.0\textwidth]{out/pdf/svg/mem_management_2.pdf}}%
    \only<3>{\includegraphics[width=1.0\textwidth]{out/pdf/svg/mem_management_3.pdf}}%
    \only<4>{\includegraphics[width=1.0\textwidth]{out/pdf/svg/mem_management_4.pdf}}%
    \only<5>{\includegraphics[width=1.0\textwidth]{out/pdf/svg/mem_management_5.pdf}}%
    \only<6>{\includegraphics[width=1.0\textwidth]{out/pdf/svg/mem_management_6.pdf}}%
    \only<7>{\includegraphics[width=1.0\textwidth]{out/pdf/svg/mem_management_7.pdf}}%
    \only<8>{\includegraphics[width=1.0\textwidth]{out/pdf/svg/mem_management_8.pdf}}%
    \only<9>{\includegraphics[width=1.0\textwidth]{out/pdf/svg/mem_management_9.pdf}}%
    \only<10>{\includegraphics[width=1.0\textwidth]{out/pdf/svg/mem_management_10.pdf}}%
  \end{center}
\end{frame}


%%%%%%%%%%%%%%%%% 
\begin{frame}
  \frametitle{ページ(スワップ)アウト・イン}
  \begin{itemize}
  \item 実際にアクセスが起きたページに物理メモリを割当(要求時ページング)
  \item $\Rightarrow$ 当然, 物理メモリが足りなく(逼迫する)ことがあり得る
  \item OSは物理メモリが逼迫すると,
    一部の内容を2次記憶に退避して, 解放する(\ao{\bf ページアウト, スワップアウト})
  \item 退避されたページの, ページテーブルエントリは「物理ページ不在」とし,
    次にアクセスする際にページフォルトが起きるようにしておく
  \item ページアウトされた領域が再びアクセスされたら,
    ページフォルト処理で, 2次記憶に退避された内容を読み込む(Unix: \ao{\bf メジャー}フォルト)
  \end{itemize}
\end{frame}

%%%%%%%%%%%%%%%%% 
\begin{frame}
  \frametitle{マイナーフォルト vs. メジャーフォルト}
  \begin{itemize}
  \item マイナーフォルト: 割当てられたページへの\ao{初めての}アクセス時におこるページフォルト
    \begin{itemize}
    \item 要求時ページングをしている限り必然的に(1ぺーじにつき1度は)起こる
    \item 実際の処理は物理メモリの探索 $+$ ゼロ初期化
    \end{itemize}
  \item メジャーフォルト: ページアウトされたページに対するアクセス時におこるページフォルト
    \begin{itemize}
    \item ページの内容を2次記憶から読み込む処理を伴う
    \item マイナーフォルトより時間がかかる
    \end{itemize}
  \end{itemize}
\end{frame}

%%%%%%%%%%%%%%%%% 
\begin{frame}
  \frametitle{メモリ管理動作図(ページアウト・ページイン)}
  \begin{center}
    \only<1>{\includegraphics[width=1.0\textwidth]{out/pdf/svg/mem_management_11.pdf}}%
    \only<2>{\includegraphics[width=1.0\textwidth]{out/pdf/svg/mem_management_12.pdf}}%
    \only<3>{\includegraphics[width=1.0\textwidth]{out/pdf/svg/mem_management_13.pdf}}%
    \only<4>{\includegraphics[width=1.0\textwidth]{out/pdf/svg/mem_management_14.pdf}}%
    \only<5>{\includegraphics[width=1.0\textwidth]{out/pdf/svg/mem_management_15.pdf}}%
    \only<6>{\includegraphics[width=1.0\textwidth]{out/pdf/svg/mem_management_16.pdf}}%
  \end{center}
\end{frame}

%%%%%%%%%%%%%%%%%
\section{資源使用量を知る・制限するAPI}
%%%%%%%%%%%%%%%%%

%%%%%%%%%%%%%%%%% 
\begin{frame}
  \frametitle{資源使用量を知るAPI}
  \begin{itemize}
  \item \ao{\tt int $e$ = getrusage(RUSAGE\_SELF, {\it ru});}
    \begin{itemize}
    \item 呼び出したプロセスの種々の資源使用量, イベント数を{\it ru}に返す
    \item マイナーフォルト, メジャーフォルト,
      最大物理メモリ使用量(RSS), CPU仕様時間, etc.
    \end{itemize}
  \item \ao{\tt int $e$ = getrusage(RUSAGE\_CHILDREN, {\it ru});}
    \begin{itemize}
    \item 上記同様. ただし子プロセスに対して
    \end{itemize}
  \end{itemize}
\end{frame}

    
%%%%%%%%%%%%%%%%% 
\begin{frame}
  \frametitle{資源使用量を制限するAPI}
  \begin{itemize}
  \item \ao{\tt int $e$ = setrlimit(RLIMIT\_{\it res}, {\it rl});}
    \begin{itemize}
    \item 呼び出したプロセスの資源{\it res}の使用量を制限
    \item 仮想メモリ(\ao{\tt AS}), CPU使用時間(\ao{\tt CPU}), etc.
    \item 物理メモリ(\ao{\tt RSS})は効き目がない模様
      (man page: {\it{\scriptsize ``This limit has effect only in Linux 2.4.x, x $<$ 30,
        and there affects only calls to madvise(2) specifying MADV\_WILLNEED''}})
    \end{itemize}
    
  \item \ao{\tt int $e$ = prlimit({\it pid}, RLIMIT\_{\it res}, {\it rl});}
    \begin{itemize}
    \item {\tt getrlimit, setrlimit}の機能をひとつに統合
    \item (権限があれば)制限するプロセスを指定可能
    \end{itemize}

  \item cgroups (後述)
  \end{itemize}
\end{frame}

%%%%%%%%%%%%%%%%%
\begin{frame}
  \frametitle{物理メモリの割当状況を知るAPI}
  \begin{itemize}
  \item \ao{\tt int $e$ = mincore({\it addr}, {\it len}, {\it vec});}
    \begin{itemize}
    \item $[{\it addr}, {\it addr+len})$内の各論理ページ
      が物理メモリ上にある・ないを{\it vec}から始まる(charの)配列に格納する
    \item $i$ページ目が物理メモリにある$\iff$ {\tt vec[$i$] == 1}
    \end{itemize}
  \item 本来アプリケーションが知る必要がない情報ではあるが,
    OSの挙動を知るために有用
  \end{itemize}
\end{frame}

%%%%%%%%%%%%%%%%%
\section{cgroups}
%%%%%%%%%%%%%%%%% 

%%%%%%%%%%%%%%%%% 
\begin{frame}
  \frametitle{cgroups}
  \begin{itemize}
  \item Linuxで一部のプロセス(とその子孫)に対する資源割当量を制御する仕組み
    \begin{itemize}
    \item コンテナ (Dockerなど) の基盤
    \end{itemize}
  \item 詳細 $\rightarrow$ {\tt man cgroups}
  \item version v1 と v2 がある
  \item 演習環境では cgroups v2 を設定している
  \item 以下は自分のマシンで cgroups v2 を使ってみたい人のための参考情報
  \end{itemize}
\end{frame}

%%%%%%%%%%%%%%%%% 
\begin{frame}[fragile]
  \frametitle{cgroups v2を手動で設定}
  \begin{enumerate}
  \item 適当な空ディレクトリを作成
\begin{lstlisting}
$ mkdir ~/tmp/cg2
\end{lstlisting} %$
  \item cgroup2 ファイルシステムをマウント
\begin{lstlisting}
$ sudo mount -t cgroup2 none ~/tmp/cg2
\end{lstlisting} %$
\end{enumerate}
\begin{itemize}
\item 起動時に, {\tt /sys/fs/cgroup}下にすでに設定がなされている可能性あり
\begin{lstlisting}
$ mount | grep cgroup
cgroup2 on /sys/fs/cgroup type cgroup2 (rw,nosuid,nodev,noexec,relatime)
\end{lstlisting} %$
以下が起動時になされている
\begin{lstlisting}
sudo mount -t cgroup2 /sys/fs/cgroup
\end{lstlisting}
\end{itemize}
\end{frame}

%%%%%%%%%%%%%%%%% 
\begin{frame}[fragile]
  \frametitle{起動時にcgroups v2を有効にする}
  \begin{enumerate}
  \item {\tt /etc/default/grub}で{\tt GRUB\_CMDLINE\_LINUX\_DEFAULT}を設定
\begin{lstlisting}
GRUB_CMDLINE_LINUX_DEFAULT="... @{\tt systemd.unified\_cgroup\_hierarchy=1}@"
\end{lstlisting}
\item 上記設定をカーネル起動パラメータに反映
\begin{lstlisting}
$ sudo update-grub
\end{lstlisting} %$
\item 再起動
\end{enumerate}

\begin{itemize}
\item 以下のファイルが見えれば(*)有効
\begin{lstlisting}
$ ls /sys/fs/cgroup
... cgroup.procs ... cgroup.subtree_control ...
\end{lstlisting} %$

\item 以下はこうなっている前提で説明
\end{itemize}
\end{frame}

%%%%%%%%%%%%%%%%% 
\begin{frame}[fragile]
\frametitle{cgroups v2の基本操作}
\begin{enumerate}
\item トップディレクトリ({\tt /sys/fs/cgroup})下に好きなディレクトリ階層構造を作る
\begin{lstlisting}
$ @\underline{\tt sudo mkdir /sys/fs/cgroup/{\it R}}@
$ @\underline{\tt sudo mkdir /sys/fs/cgroup/{\it R}/{\it A}}@
$ @\underline{\tt sudo mkdir /sys/fs/cgroup/{\it R}/{\it B}}@
$ ...
\end{lstlisting} %$
\begin{itemize}
\item ディレクトリ$\leftrightarrow$プロセスのグループ
\end{itemize}
\item プロセスのグループに対し資源量(物理メモリ量など)の制限をかける
\begin{lstlisting}
# @{\it R}@下のグループの memory を制御
$ @\underline{\tt echo +memory | sudo tee /sys/fs/cgroup/{\it R}/cgroup.subtree\_control}@
# @{\it A}@に属するプロセスグループの合計物理メモリ使用量を256MBにする
$ @\underline{\tt echo 256M | sudo tee /sys/fs/cgroup/{\it R}/{\it A}/memory.high}@
\end{lstlisting} %$
\item プロセスをグループに所属させる
\begin{lstlisting}
$ @\underline{\tt echo {\it pid} | sudo tee /sys/fs/cgroup/{\it R}/{\it A}/cgroup.procs}@
\end{lstlisting} %$
\end{enumerate}
\end{frame}

%%%%%%%%%%%%%%%%% 
\begin{frame}[fragile]
  \frametitle{演習用Jupyter環境 cg\_mem\_limit の正体}
  \begin{itemize}
  \item コマンドの仕様
\begin{lstlisting}
cg_mem_limit @{\it コマンド}@      
\end{lstlisting}
で{\it コマンド}の物理メモリ使用量を256MBに制限 (それを越えたらページアウト)
\item 仕組み
  \begin{enumerate}
  \item fork
  \item 子プロセスのIDを{\tt /sys/fs/cgroup/taulec/{\it user}/cgroup.procs}に書く
  \item 子プロセスはそれを待ってからexec
  \end{enumerate}
\item 注: 事前設定
  \begin{itemize}
  \item []
\begin{lstlisting}
  $ echo +memory | sudo tee /sys/fs/cgroup/taulec/cgroup.subtree_control
\end{lstlisting}%$
\item 全userに対し
\begin{lstlisting}
$ mkdir /sys/fs/cgroup/taulec/@{\it user}@
$ echo 256M | sudo tee /sys/fs/cgroup/taulec/@{\it user}@/memory.high
\end{lstlisting}%$
\end{itemize}
\end{itemize}
\end{frame}

%%%%%%%%%%%%%%%%% 
\begin{frame}[fragile]
  \frametitle{スワップ領域}
  \begin{itemize}
  \item memory.highを設定する(物理メモリ量を制限する)場合,
    \ao{スワップ領域}の有効化が必要
  \item すでに設定されているか? $\rightarrow$ {\tt free, swapon}コマンド
\begin{lstlisting}
$ @\underline{\tt free}@
       total     used     free      shared  buff/cache available
Mem:   32513016  5507528  20926012  124652  6079476    26408444
@\ao{\tt Swap:\ \ \ 2097148\ \ \ \ \ 6144\ \ \ 2091004}@
$ @\underline{\tt swapon}@
NAME      TYPE SIZE USED PRIO
@\ao{\tt /swapfile file\ \ \ 2G\ \ \ 6M\ \ \ -2}@
\end{lstlisting} %$
\end{itemize}
\end{frame}

%%%%%%%%%%%%%%%%% 
\begin{frame}[fragile]
  \frametitle{スワップ領域の有効化}
  \begin{enumerate}
  \item 大きさに合わせたファイルをつくる (2048MBの例)
\begin{lstlisting}
$ sudo dd if=/dev/zero of=/swapfile bs=$((1024 * 1024)) count=2048
\end{lstlisting}
\item スワップ領域用のファイルとして初期化
\begin{lstlisting}
$ sudo chmod 600 /swapfile
$ sudo mkswap /swapfile
\end{lstlisting} %$
\item 有効化する
\begin{lstlisting}
$ sudo swapon /swapfile
\end{lstlisting} %$
\end{enumerate}
\begin{itemize}
\item [] 設定を永続化したければ以下を{\tt /etc/fstab}に記述
\begin{lstlisting}
/swapfile swap swap defaults 0 0
\end{lstlisting}
\end{itemize}
\end{frame}

%%%%%%%%%%%%%%%%%
\section{ページ置換アルゴリズム}
%%%%%%%%%%%%%%%%%

%%%%%%%%%%%%%%%%% 
\begin{frame}
  \frametitle{ページ置換アルゴリズム}
  \begin{itemize}
  \item \ao{問題:} 物理メモリ上のページを2次記憶に退避(ページアウト)する際,
    どのページを退避させるか?
    \only<1>{\includegraphics[width=0.8\textwidth]{out/pdf/svg/mem_management_12.pdf}}%
    \only<2>{\includegraphics[width=0.8\textwidth]{out/pdf/svg/mem_management_13.pdf}}
  \item \ao{目標:} ページの置換(退避と復帰)のコストの最小化
  \item $\rightarrow$ \ao{ページ置換アルゴリズム}
  \end{itemize}
\end{frame}

%%%%%%%%%%%%%%%%% 
\begin{frame}
  \frametitle{ページ置換の問題のモデル化 (入力)}
  \begin{itemize}
  \item 入力
    \begin{itemize}
    \item $\ao{L}$ : アクセスされるページ数 (ページ番号 $0, \cdots, L-1$)
    \item $\ao{P}$ : 物理メモリサイズ (ページ数) 
    \item $\ao{R_0}$ : 初期常駐ページ集合 ($\ao{|R_0| = P}$)
    \item $\ao{\boldsymbol{a}} = a_0, a_1, a_2, \cdots , a_{n-1}$ : アクセスされるページの系列
    \end{itemize}
  \end{itemize}

  \begin{center}
    %\includegraphics[width=0.8\textwidth]{out/pdf/svg/online_offline_1.pdf}
    \def\svgwidth{\textwidth}
    {\scriptsize\input{out/tex/svg/online_offline_1.pdf_tex}}
  \end{center}
\end{frame}

%%%%%%%%%%%%%%%%% 
\begin{frame}
  \frametitle{ページ置換の問題のモデル化 (出力)}
  \begin{itemize}
  \item 出力
    \begin{itemize}
    \item 各アクセス後の常駐ページ集合の系列
      $\ao{{\boldsymbol{R}} = R_1, R_2, \cdots, R_n}$
    \item ただし以下を条件とする
      \begin{itemize}
      \item $|R_i| = P$ 
      \item $a_i \in R_{i+1}$
        (解釈: $a_i$アクセス直後は$a_i$が物理メモリ上にある)
      \end{itemize}
    \end{itemize}
  \end{itemize}

  \begin{center}
    %\includegraphics[width=0.8\textwidth]{out/pdf/svg/online_offline_2.pdf}
    \def\svgwidth{\textwidth}
    {\scriptsize\input{out/tex/svg/online_offline_2.pdf_tex}}
  \end{center}
\end{frame}
    
%%%%%%%%%%%%%%%%% 
\begin{frame}
  \frametitle{ページ置換の問題のモデル化 (評価尺度)}
  \begin{itemize}
  \item 評価尺度 $=$ ページ置換数
    \[ \mbox{REP}({\boldsymbol{R}}) \equiv \ao{\sum_{i=0,1,\cdots, n-1} |R_i \setminus R_{i+1}|} \]
    \begin{itemize}
    \item $(R_i \setminus R_{i+1})$ は\ao{集合の差分} (ページアウト数)
    \item $|R_i| = P$の条件下では\ao{ページアウト数$=$ページイン数}
    \item (アクセスされていないページをインすることはないとすると) \ao{$=$ページフォルト数}
    \end{itemize}
  \end{itemize}

  \begin{center}
    %\includegraphics[width=0.9\textwidth]{out/pdf/svg/online_offline_3.pdf}
    \def\svgwidth{\textwidth}
    {\scriptsize\input{out/tex/svg/online_offline_3.pdf_tex}}
  \end{center}
\end{frame}

%%%%%%%%%%%%%%%%% 
\begin{frame}
  \frametitle{2つの問題設定}
  \begin{itemize}
  \item<1-> \ao{オフライン}問題:
    アルゴリズム($A$)は, 将来のアクセス系列を全て入力とする(知っている)
    \[ A_{\mbox{\scriptsize OFF}}(R_i, [a_0, a_1, \cdots , a_{i-1}], \ao{[a_i, \cdots , a_{n-1}}]) = R_{i+1} \]
  \item<2-> \ao{オンライン}問題:
    アルゴリズム($A$)は, 将来のアクセスはわからない
    \[ A_{\mbox{\scriptsize ON}}(R_i, [a_0, a_1, \cdots , a_{i-1}], \ao{[a_i]}) = R_{i+1} \]
  \end{itemize}

  \begin{center}
    %\only<1>{\includegraphics[width=0.7\textwidth]{out/pdf/svg/online_offline_5.pdf}}%
    %\only<2>{\includegraphics[width=0.7\textwidth]{out/pdf/svg/online_offline_4.pdf}}%
    \def\svgwidth{0.8\textwidth}
    \only<1>{{\tiny\input{out/tex/svg/online_offline_5.pdf_tex}}}%
    \only<2>{{\tiny\input{out/tex/svg/online_offline_4.pdf_tex}}}%
  \end{center}
\end{frame}

%%%%%%%%%%%%%%%%% 
\begin{frame}
  \frametitle{オフライン vs オンライン}
  \begin{itemize}
  \item \ao{オフライン}問題
    \begin{itemize}
    \item 将来のアクセスがわかっているという仮定は非現実的
    \item \ao{最適なアルゴリズムが存在}し, 現実にあわない仮定であっても設計の指針にはなる
    \end{itemize}
  \item \ao{オンライン}問題
    \begin{itemize}
    \item 問題設定は現実的
    \item \ao{どんなアルゴリズム}でも最悪ケースは同じ
    \end{itemize}
  \end{itemize}

  \begin{center}
    %\only<1>{\includegraphics[width=0.7\textwidth]{out/pdf/svg/online_offline_5.pdf}}%
    %\only<2>{\includegraphics[width=0.7\textwidth]{out/pdf/svg/online_offline_4.pdf}}%
    \def\svgwidth{0.8\textwidth}
    {\tiny\input{out/tex/svg/online_offline_4.pdf_tex}}
  \end{center}
\end{frame}

%%%%%%%%%%%%%%%%% 
\begin{frame}
  \frametitle{重要性を確認する問題}
  \begin{itemize}
  \item 初期常駐ページ $R_0 = \{0, 1, \cdots , P-1\}$ 
  \item アクセス系列
    \[ {\mathbf a} = P, 0, 1, \cdots , P, 0, 1, \cdots, P, 0, 1, \cdots, P, \cdots \]
    \begin{itemize}
    \item (物理メモリ量 + 1ページ)を何度もスキャン
      
    \item $\approx$ 物理メモリを上回る配列の全てを何度も同じ順番でアクセス
    \end{itemize}

    \begin{center}
      \includegraphics[width=0.4\textwidth]{out/pdf/svg/page_replacement_1.pdf}
    \end{center}
  \end{itemize}
\end{frame}

%%%%%%%%%%%%%%%%% 
\begin{frame}
  \frametitle{最悪 vs. 最適}
  ページ$q$をアクセスしてページアウトが必要になったとき, 
  
  \begin{columns}
    \begin{column}{0.5\textwidth}
      \begin{itemize}
      \item 最悪
        \[ \aka{(q+1)} \mod (P+1) \]
        をページアウト
      \item<8-> ページフォルト率 \aka{1}
      \end{itemize}
    \end{column}
    \begin{column}{0.5\textwidth}
      \begin{itemize}
      \item 最適(と思われる)アルゴリズム
        \[ \ao{(q-1)} \mod (P+1) \]
        をページアウト
      \item<17-> ページフォルト率 \ao{$1/P$}
      \end{itemize}
    \end{column}
  \end{columns}
  
  \begin{columns}
    \begin{column}{0.5\textwidth}
      \begin{center}
        \only<1>{\includegraphics[width=0.6\textwidth]{out/pdf/svg/page_replacement_2.pdf}}%
        \only<2>{\includegraphics[width=0.6\textwidth]{out/pdf/svg/page_replacement_3.pdf}}%
        \only<3>{\includegraphics[width=0.6\textwidth]{out/pdf/svg/page_replacement_4.pdf}}%
        \only<4>{\includegraphics[width=0.6\textwidth]{out/pdf/svg/page_replacement_5.pdf}}%
        \only<5>{\includegraphics[width=0.6\textwidth]{out/pdf/svg/page_replacement_6.pdf}}%
        \only<6>{\includegraphics[width=0.6\textwidth]{out/pdf/svg/page_replacement_7.pdf}}%
        \only<7->{\includegraphics[width=0.6\textwidth]{out/pdf/svg/page_replacement_8.pdf}}%
      \end{center}
    \end{column}
    \begin{column}{0.5\textwidth}
      \begin{center}
        \only<1-8>{\includegraphics[width=0.6\textwidth]{out/pdf/svg/page_replacement_2.pdf}}%
        \only<9>{\includegraphics[width=0.6\textwidth]{out/pdf/svg/page_replacement_3.pdf}}%
        \only<10>{\includegraphics[width=0.6\textwidth]{out/pdf/svg/page_replacement_9.pdf}}%
        \only<11>{\includegraphics[width=0.6\textwidth]{out/pdf/svg/page_replacement_10.pdf}}%
        \only<12>{\includegraphics[width=0.6\textwidth]{out/pdf/svg/page_replacement_11.pdf}}%
        \only<13>{\includegraphics[width=0.6\textwidth]{out/pdf/svg/page_replacement_12.pdf}}%
        \only<14>{\includegraphics[width=0.6\textwidth]{out/pdf/svg/page_replacement_13.pdf}}%
        \only<15>{\includegraphics[width=0.6\textwidth]{out/pdf/svg/page_replacement_14.pdf}}%
        \only<16->{\includegraphics[width=0.6\textwidth]{out/pdf/svg/page_replacement_15.pdf}}%
      \end{center}
    \end{column}
  \end{columns}
\end{frame}

%%%%%%%%%%%%%%%%% 
\begin{frame}
  \frametitle{実践的な注}
  \begin{itemize}
  \item ページ置換が重要とは言え,
    実践的には, ページ置換は頻繁に起きてはならない
  \item 起きているのならメモリが足りない
  \item ページ置換が頻繁に起きる状態を
    \aka{スラッシング}といい,
    メモリが足りない
    (またはアプリがメモリを使いすぎている)徴候
  \end{itemize}
\end{frame}

%%%%%%%%%%%%%%%%% 
\begin{frame}
  \frametitle{オフライン問題に対する最適アルゴリズム}
  \begin{itemize}
  \item 最適アルゴリズム(B\'{e}l\'{a}dyのOPT): 状態$R_i$で$a_i$がアクセスされたとき,
    \begin{itemize}
    \item $a_i \in R_i \Rightarrow$ 何もしない $\iff R_{i+1} = R_i$  (当然)
    \item $a_i \not\in R_i \Rightarrow$ 常駐ページのうち,
      \ao{次にアクセスされるまでの時間が最も長い}ページを置換する
      \ao{(次におこるページ置換を極力先延ばしにする)}
    \end{itemize}
  \item 例
    \begin{eqnarray*}
      & & A_{\mbox{\scriptsize OPT}}(\{1,3,7,9\},\ast, [\ao{2},6,\underline{1},6,\underline{9},\underline{7},4,1,5,\underline{\aka{3}},3,5\cdots]) \\
      & = & \{1,3,7,9\} \setminus \{\aka{3}\} \cup \{\ao{2}\} \\
      & = & \{1,2,7,9\}
    \end{eqnarray*}

  \item 前掲「重要性を確認する問題」に適用してみよ
  \end{itemize}
\end{frame}

%%%%%%%%%%%%%%%%% 
\begin{frame}
  \frametitle{証明のヒント}
  \begin{itemize}
  \item 最適アルゴリズム $A_{\mbox{\scriptsize OPT}}$ の系列
    \[ \boldsymbol{R} = R_0 \stackrel{a_0}{\longrightarrow} R_{1} \stackrel{a_{1}}{\longrightarrow} \cdots \stackrel{a_{n}}{\longrightarrow} R_n \]
  \item[] 任意のアルゴリズムの系列
    \[ \boldsymbol{S} = S_0 \stackrel{a_0}{\longrightarrow} S_{1} \stackrel{a_{1}}{\longrightarrow} \cdots \stackrel{a_{n}}{\longrightarrow} S_n \]
  \item[] に対し以下を$n$ (系列の長さ)に関する帰納法で
    \[ \mbox{REP}(\boldsymbol{R}) \leq \mbox{REP}(\boldsymbol{S}) + |S_0 - R_0| \quad (\dagger) \]
  \item これが言えれば$S_0 = R_0$とすることで証明が完了
  \end{itemize}
\end{frame}

%%%%%%%%%%%%%%%%% 
\begin{frame}
  \frametitle{オンライン問題に対してはどんなアルゴリズムの最悪ケースも同じ}
  \begin{itemize}
  \item 明らかに, どんなアルゴリズムも, $P < L$である限り,
    物理メモリ上にないページが直後にアクセスされることを避けられない
  \item 何をやっても「最悪のケース(ページフォルト率1)」は避けられない
  \end{itemize}
\end{frame}

%%%%%%%%%%%%%%%%% 
\begin{frame}
  \frametitle{実際のアルゴリズム}
  \begin{itemize}
  \item オフラインの最適アルゴリズムは無理, オンラインの最適アルゴリズムは考えるだけ無駄
  \item どうするか?
    \begin{itemize}
    \item \ao{「次に使われるのがいつか(すぐか, 当分先か)」の予想}に基づく
    \item それに基づき, すぐ使われる(可能性が高そうな)ものを残す
    \end{itemize}
  \item どう予想するか?
    \begin{itemize}
    \item 経験則: \ao{最近使われたものはまたすぐ使われる}
    \end{itemize}
  \item $\Rightarrow$ アルゴリズム例
    \begin{enumerate}
    \item LRU
    \item FIFO
    \item LRUの近似 (セカンドチャンスまたはクロック)
    \item エイジング
    \end{enumerate}
  \end{itemize}
\end{frame}

%%%%%%%%%%%%%%%%% 
\begin{frame}
  \frametitle{LRU置換}
  \begin{itemize}
  \item ページアウト時に,
    \ao{Least Recently Used (過去にアクセスされた時点が最も遠い)な}ページを置換する
    \begin{itemize}
    \item しばらく使われていないページは今後も当分使われない
    \item $\approx$
      最近使われたページは近い将来またすぐ使われる
    \end{itemize}
  \item 根拠となるプログラムの性質: \ao{\bf アクセスの局所性}
    \begin{itemize}
    \item \ao{\bf 空間的}局所性 : あるアドレスをアクセスした直後,
      近い(例: 同じページの別の)アドレスをアクセスする傾向
    \item \ao{\bf 時間的}局所性 : あるアドレスをアクセスしてからあまり間をおかずに,
      同じアドレスをアクセスする傾向
    \end{itemize}
  \item LRU置換は,
    未来のアクセスが過去の反転であるとして, $A_{\mbox{\scriptsize OPT}}$を適用しているのと同じ
    \begin{eqnarray*}
      &  & A_{\mbox{\scriptsize LRU}}(R, [a_0, \cdots , a_{i-1}], [a_i]) \\
      & = & A_{\mbox{\scriptsize OPT}}(R, [a_0, \cdots , a_{i-2}, a_{i-1}], [a_i, a_{i-1}, a_{i-2}, \cdots])
    \end{eqnarray*}
  \end{itemize}
\end{frame}

%%%%%%%%%%%%%%%%% 
\begin{frame}
  \frametitle{若干の脱線}
  \begin{itemize}
  \item LRU置換は多くの現実的なアクセスパターンで良い性能を示すが,
    「配列のスキャンの繰り返し」というよくあるパターンで,
    「配列のサイズ $>$ 物理メモリ量」となった途端に最悪の性能を示す
  \item 前掲「重要性を確認する問題」で,
    LRU置換が最悪の性能(ページフォルト率1)になることを確認せよ
    \begin{center}
      \only<1>{\includegraphics[width=0.4\textwidth]{out/pdf/svg/page_replacement_2.pdf}}%
      \only<2>{\includegraphics[width=0.4\textwidth]{out/pdf/svg/page_replacement_3.pdf}}%
      \only<3>{\includegraphics[width=0.4\textwidth]{out/pdf/svg/page_replacement_4.pdf}}%
      \only<4>{\includegraphics[width=0.4\textwidth]{out/pdf/svg/page_replacement_5.pdf}}%
      \only<5>{\includegraphics[width=0.4\textwidth]{out/pdf/svg/page_replacement_6.pdf}}%
      \only<6>{\includegraphics[width=0.4\textwidth]{out/pdf/svg/page_replacement_7.pdf}}%
      \only<7->{\includegraphics[width=0.4\textwidth]{out/pdf/svg/page_replacement_8.pdf}}%
    \end{center}
  \end{itemize}
\end{frame}

%%%%%%%%%%%%%%%%% 
\begin{frame}
  \frametitle{LRU実装に必要なデータ構造}
  \begin{itemize}
  \item 物理メモリ上の$P$ページを「LRU順 (先頭がLeast, 末尾がMost Recently Used)」
    に保持 \ao{(LRUリスト)}
    \begin{itemize}
    \item 物理メモリ上にないページへのアクセス(ページフォルト発生)時
      \begin{itemize}
      \item リスト先頭(図の左端)のページを除去
      \item アクセスされたページは末尾へ挿入
      \end{itemize}
    \item 物理メモリ上にあるページへのアクセス時
      \begin{itemize}
      \item そのページをリストの末尾(図の右端)へ移動
      \end{itemize}
    \end{itemize}
  \begin{center}
\only<1>{\includegraphics[width=0.9\textwidth]{out/pdf/svg/lru_1.pdf}}%
\only<2>{\includegraphics[width=0.9\textwidth]{out/pdf/svg/lru_2.pdf}}%
\only<3>{\includegraphics[width=0.9\textwidth]{out/pdf/svg/lru_3.pdf}}%
\only<4>{\includegraphics[width=0.9\textwidth]{out/pdf/svg/lru_4.pdf}}%
\only<5>{\includegraphics[width=0.9\textwidth]{out/pdf/svg/lru_5.pdf}}%
\only<6>{\includegraphics[width=0.9\textwidth]{out/pdf/svg/lru_6.pdf}}%
\only<7>{\includegraphics[width=0.9\textwidth]{out/pdf/svg/lru_7.pdf}}%
\only<8>{\includegraphics[width=0.9\textwidth]{out/pdf/svg/lru_8.pdf}}%
\only<9>{\includegraphics[width=0.9\textwidth]{out/pdf/svg/lru_9.pdf}}%
\only<10>{\includegraphics[width=0.9\textwidth]{out/pdf/svg/lru_10.pdf}}%
\only<11>{\includegraphics[width=0.9\textwidth]{out/pdf/svg/lru_11.pdf}}%
  \end{center}
  \end{itemize}
\end{frame}

%%%%%%%%%%%%%%%%% 
\begin{frame}
  \frametitle{LRU実装の困難さ}
  \begin{itemize}
  \item 物理メモリ上にあるページへのアクセスの度にデータ構造を変更する必要がある
    \begin{itemize}
    \item 例外は発生しないので, ソフトウェアで介入は困難
    \item 特別なハードウェアの仕組みが必要(メモリアクセスのたびに介入)
    \end{itemize}
  \item $\Rightarrow$
    LRU置換の近似 (\ao{Not Recently Used (NRU)} 置換)
  \item $\approx$ 「最近使われたかどうか」を大雑把に把握する
  \end{itemize}
\end{frame}

%%%%%%%%%%%%%%%%% 
\begin{frame}
  \frametitle{FIFO}
  \begin{itemize}
  \item 物理メモリ上の$P$ページを「ページインされた順」
    に保持 \ao{(FIFO)}
  \item 物理メモリ上にないページへのアクセス(ページフォルト発生)時
    \begin{itemize}
    \item LRUと同じ
    \end{itemize}
  \item 物理メモリ上にあるページへのアクセス時
    \begin{itemize}
    \item \ao{何もしない}
    \end{itemize}
  \end{itemize}
  
  \begin{center}
\only<1>{\includegraphics[width=0.9\textwidth]{out/pdf/svg/lru_1.pdf}}%
\only<2>{\includegraphics[width=0.9\textwidth]{out/pdf/svg/lru_2.pdf}}%
\only<3>{\includegraphics[width=0.9\textwidth]{out/pdf/svg/lru_3.pdf}}%
\only<4>{\includegraphics[width=0.9\textwidth]{out/pdf/svg/lru_4.pdf}}%
\only<5>{\includegraphics[width=0.9\textwidth]{out/pdf/svg/lru_5.pdf}}%
\only<6->{\includegraphics[width=0.9\textwidth]{out/pdf/svg/lru_6.pdf}}%
  \end{center}
    
  \begin{itemize}
  \item<7-> \aka{問題点:} ページインしてから使われたものとそうでないものを区別できない
  \item<8> $\Rightarrow$ \ao{セカンドチャンス}または\ao{クロックアルゴリズム}
  \end{itemize}
\end{frame}

%%%%%%%%%%%%%%%%% 
\begin{frame}
  \frametitle{セカンドチャンス}
  \begin{itemize}
  \item 物理メモリ上の$P$ページを「ページインされた順」に保持 \ao{(FIFO)}
  \item 物理メモリ上にあるページへのアクセス時
    \begin{itemize}
    \item \ao{(MMUが)そのページの\ao{参照ビット}を1にする}
    \end{itemize}
  \item 物理メモリ上にないページへのアクセス(ページフォルト発生)時;
    リスト先頭のページの参照ビットが
    \begin{itemize}
    \item \ao{$=0$}ならばページアウト
    \item \ao{$=1$}ならば, 0にしてリスト末尾へ再挿入(復活のチャンス, 執行猶予)
    \end{itemize}
  \end{itemize}
  \begin{center}
\only<1>{\includegraphics[width=0.9\textwidth]{out/pdf/svg/lru_13.pdf}}%
\only<2>{\includegraphics[width=0.9\textwidth]{out/pdf/svg/lru_14.pdf}}%
\only<3>{\includegraphics[width=0.9\textwidth]{out/pdf/svg/lru_15.pdf}}%
\only<4>{\includegraphics[width=0.9\textwidth]{out/pdf/svg/lru_16.pdf}}%
\only<5>{\includegraphics[width=0.9\textwidth]{out/pdf/svg/lru_17.pdf}}%
\only<6>{\includegraphics[width=0.9\textwidth]{out/pdf/svg/lru_18.pdf}}%
\only<7>{\includegraphics[width=0.9\textwidth]{out/pdf/svg/lru_19.pdf}}%
\only<8>{\includegraphics[width=0.9\textwidth]{out/pdf/svg/lru_20.pdf}}%
\only<9>{\includegraphics[width=0.9\textwidth]{out/pdf/svg/lru_21.pdf}}%
  \end{center}
\end{frame}

%%%%%%%%%%%%%%%%% 
\begin{frame}
  \frametitle{クロックアルゴリズム}
  \begin{itemize}
  \item 動作はセカンドチャンスと同じ
  \item 二重リンクリストの代わりに, 循環バッファ(配列)を使う
    (途中から要素を削除することはないのでそれで十分)
  \item 詳細は省略(データ構造の練習問題)
  \end{itemize}
\end{frame}

%%%%%%%%%%%%%%%%% 
\begin{frame}
  \frametitle{セカンドチャンスの性質}
  \begin{itemize}
  \item 参照ビットは,
    「最近アクセスがあったか否か」の指標

    \includegraphics[width=0.9\textwidth]{out/pdf/svg/lru_14.pdf}

  \item LRUリストの先頭から$i$番目(左端が0番目)のページの参照ビットが1
    $\iff$ $(P-i)$回前(直前が1回前)
    のページフォルト時またはそれ以降にアクセスされている $\quad (\dagger)$
    \begin{itemize}
    \item 注: ($\dagger$)はページフォルト時に「物理メモリ上の全ページの参照ビットが1」
      というケースを考えるとは少し不正確だが, 本質的ではないので深入りしない
    \end{itemize}
  \end{itemize}
\end{frame}

%%%%%%%%%%%%%%%%% 
\begin{frame}
  \frametitle{セカンドチャンスの限界}
  \begin{itemize}
  \item $P$回のページフォルトより短い粒度や, それ以前のアクセス履歴は用いていない
  \item 例えば以下のようなアクセス履歴を持つページ($p$, $q$, $r$)を区別できない

    \begin{center}
    \def\svgwidth{\textwidth}
    {\tiny\input{out/tex/svg/second_chance_limitation_1.pdf_tex}}
    \end{center}
  \end{itemize}
\end{frame}

%%%%%%%%%%%%%%%%% 
\begin{frame}
  \frametitle{エイジング}
  \begin{itemize}
  \item 物理メモリ上の各ページ$p$に, 一定間隔の区間ごとの「アクセスあり・なし」を,
    $N$世代分記録
    したカウンタ値$H_p$を持たせる

    \begin{center}
    \def\svgwidth{0.5\textwidth}
    \only<1>{{\tiny\input{out/tex/svg/aging_1.pdf_tex}}}%
    \only<2>{{\tiny\input{out/tex/svg/aging_2.pdf_tex}}}%
    \only<3>{{\tiny\input{out/tex/svg/aging_3.pdf_tex}}}%
    \end{center}
  \item 「1世代」$=$一定の実時間かページ置換回数
  \item<2-> 世代が変わる時, $H_p$をdecayさせ, 直近のアクセスある・なしを反映
    \[\left\{\begin{array}{rcl}
             H_p & = & (H_p >> 1) + (R_p {\tt <<} (N-1)) \\
             R_p & = & 0
    \end{array}\right.\]
  \item<2-> $R_p$ : $p$の参照ビット
  \item<3-> ページ置換時: カウンタ値$H_p$が最小のページを退避
  \end{itemize}
\end{frame}

%%%%%%%%%%%%%%%%% 
\begin{frame}
  \frametitle{現実の実装における考慮}
現実の実装ではこれまでの考察に加え以下を考慮する
\begin{itemize}
\item ページアウトは物理メモリが完全に満杯する前に(大部分埋まったところで),
  ページフォルトと非同期に始める
\item ページアウト対象を選ぶ際は, ページが最近アクセスされているかだけではなく,
  2次記憶への書き込みが必要かも考慮する
  \begin{itemize}
  \item あるページが初めてページアウトされるとき$\Rightarrow$書き込み必要
  \item あるページがページインしてから変更されているとき$\Rightarrow$書き込み必要
  \item あるページがページインしてから変更されていないとき$\Rightarrow$書き込み不要
    (単に物理メモリを開放すれば良い)
  \end{itemize}
\item ページアウトは1ページ毎ではなく, 連続した数十ページをまとめて行った方が,
  IO性能の効率がよい
\end{itemize}
\end{frame}

%%%%%%%%%%%%%%%%% 
\begin{frame}
  \frametitle{ページングを制御するAPI}
  \begin{itemize}
  \item \ao{\tt int $e$ = posix\_madvise({\it addr}, {\it length}, {\it adv});}
  \item \ao{\tt int $e$ = madvise({\it addr}, {\it length}, {\it adv});}
  \item 両者は似ているが後者は Linux 特有であり, 機能が多い
  \item {\it adv}に [{\it addr}, {\it addr+length}) の範囲のアクセスパターンを教える
    \begin{itemize}
    \item {\tt (POSIX\_)MADV\_NORMAL}
    \item {\tt (POSIX\_)MADV\_SEQUENTIAL} : 逐次的にアクセスする
    \item {\tt (POSIX\_)MADV\_RANDOM} : ランダムにアクセスする
    \item {\tt (POSIX\_)MADV\_WILLNEED} : 近い将来必要
    \item {\tt POSIX\_MADV\_DONTNEED} : 当分不要
    \item {\tt MADV\_DONTNEED} : 不要 (値が失われても良い)
    \item {\tt MADV\_COLD} : ページアウト候補対象に
    \item \ao{\tt MADV\_PAGEOUT} : 即座にページアウト
    \end{itemize}
  \end{itemize}
\end{frame}

%%%%%%%%%%%%%%%%% 
\begin{frame}
  \frametitle{madviseの利用法}
  \begin{itemize}
  \item OSが必要とするのはどれをページアウトすべきか
  \item 特に{\tt MADV\_PAGEOUT}は直接それをアプリケーションからOSに教えることを可能にする
  \item アプリケーションが今後のアクセスパターンを知っていれば,
    $A_{\mbox{\scriptsize OPT}}$に従って, ページアウト対象を決めれば良い
  \item それ以外の{\it adv}の効果は不明
  \end{itemize}
\end{frame}

\end{document}
