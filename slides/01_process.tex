\documentclass[12pt,dvipdfmx]{beamer}
\institute{}
\author{田浦健次朗}
\date{}

\usepackage{graphicx}
\DeclareGraphicsExtensions{.pdf}
\DeclareGraphicsExtensions{.eps}
\graphicspath{{out/}{out/tex/}{out/tex/gpl/}{out/tex/svg/}{out/tex/lsvg/}{out/tex/dot/}}
% \graphicspath{{out/}{out/tex/}{out/pdf/}{out/eps/}{out/tex/gpl/}{out/tex/svg/}{out/pdf/dot/}{out/pdf/gpl/}{out/pdf/img/}{out/pdf/odg/}{out/pdf/svg/}{out/eps/dot/}{out/eps/gpl/}{out/eps/img/}{out/eps/odg/}{out/eps/svg/}}
\usepackage{listings,jlisting}
\usepackage{fancybox}
\usepackage{hyperref}
\usepackage{color}

%%%%%%%%%%%%%%%%%%%%%%%%%%%
%%% themes
%%%%%%%%%%%%%%%%%%%%%%%%%%%
\usetheme{default} % Szeged
%% no navigation bar
% default boxes Bergen Boadilla Madrid Pittsburgh Rochester
%% tree-like navigation bar
% Antibes JuanLesPins Montpellier
%% toc sidebar
% Berkeley PaloAlto Goettingen Marburg Hannover Berlin Ilmenau Dresden Darmstadt Frankfurt Singapore Szeged
%% Section and Subsection Tables
% Copenhagen Luebeck Malmoe Warsaw

%%%%%%%%%%%%%%%%%%%%%%%%%%%
%%% innerthemes
%%%%%%%%%%%%%%%%%%%%%%%%%%%
% \useinnertheme{circles}	% default circles rectangles rounded inmargin

%%%%%%%%%%%%%%%%%%%%%%%%%%%
%%% outerthemes
%%%%%%%%%%%%%%%%%%%%%%%%%%%
% outertheme
% \useoutertheme{default}	% default infolines miniframes smoothbars sidebar sprit shadow tree smoothtree


%%%%%%%%%%%%%%%%%%%%%%%%%%%
%%% colorthemes
%%%%%%%%%%%%%%%%%%%%%%%%%%%
\usecolortheme{seahorse}
%% special purpose
% default structure sidebartab 
%% complete 
% albatross beetle crane dove fly seagull 
%% inner
% lily orchid rose
%% outer
% whale seahorse dolphin

%%%%%%%%%%%%%%%%%%%%%%%%%%%
%%% fontthemes
%%%%%%%%%%%%%%%%%%%%%%%%%%%
\usefonttheme{serif}  
% default professionalfonts serif structurebold structureitalicserif structuresmallcapsserif

%%%%%%%%%%%%%%%%%%%%%%%%%%%
%%% generally useful beamer settings
%%%%%%%%%%%%%%%%%%%%%%%%%%%
% 
\AtBeginDvi{\special{pdf:tounicode EUC-UCS2}}
% do not show navigation
\setbeamertemplate{navigation symbols}{}
% show page numbers
\setbeamertemplate{footline}[frame number]

%%%%%%%%%%%%%%%%%%%%%%%%%%%
%%% define some colors for convenience
%%%%%%%%%%%%%%%%%%%%%%%%%%%

\newcommand{\mido}[1]{{\color{green}#1}}
\newcommand{\mura}[1]{{\color{purple}#1}}
\newcommand{\ore}[1]{{\color{orange}#1}}
\newcommand{\ao}[1]{{\color{blue}#1}}
\newcommand{\aka}[1]{{\color{red}#1}}

\setbeamercolor{ex}{bg=cyan!20!white}

\definecolor{UniBlue}{RGB}{20,20,250} 
\setbeamercolor{structure}{fg=UniBlue} % 見出しカラー

\iffalse
%%%%%%%%%%%%%%%%%%%%%%%%%%%
%% customize beamer template
%% https://www.opt.mist.i.u-tokyo.ac.jp/~tasuku/beamer.html
%%%%%%%%%%%%%%%%%%%%%%%%%%%

%\renewcommand{\familydefault}{\sfdefault}  % 英文をサンセリフ体に
%\renewcommand{\kanjifamilydefault}{\gtdefault}  % 日本語をゴシック体に
\usefonttheme{structurebold} % タイトル部を太字
\setbeamerfont{alerted text}{series=\bfseries} % Alertを太字
\setbeamerfont{section in toc}{series=\mdseries} % 目次は太字にしない
\setbeamerfont{frametitle}{size=\Large} % フレームタイトル文字サイズ
\setbeamerfont{title}{size=\LARGE} % タイトル文字サイズ
\setbeamerfont{date}{size=\small}  % 日付文字サイズ

\definecolor{UniBlue}{RGB}{0,150,200} 
\definecolor{AlertOrange}{RGB}{255,76,0}
\definecolor{AlmostBlack}{RGB}{38,38,38}
\setbeamercolor{normal text}{fg=AlmostBlack}  % 本文カラー
\setbeamercolor{structure}{fg=UniBlue} % 見出しカラー
\setbeamercolor{block title}{fg=UniBlue!50!black} % ブロック部分タイトルカラー
\setbeamercolor{alerted text}{fg=AlertOrange} % \alert 文字カラー
\mode<beamer>{
    \definecolor{BackGroundGray}{RGB}{254,254,254}
    \setbeamercolor{background canvas}{bg=BackGroundGray} % スライドモードのみ背景をわずかにグレーにする
}


%フラットデザイン化
\setbeamertemplate{blocks}[rounded] % Blockの影を消す
\useinnertheme{circles} % 箇条書きをシンプルに
\setbeamertemplate{navigation symbols}{} % ナビゲーションシンボルを消す
\setbeamertemplate{footline}[frame number] % フッターはスライド番号のみ

%タイトルページ
\setbeamertemplate{title page}{%
    \vspace{2.5em}
    {\usebeamerfont{title} \usebeamercolor[fg]{title} \inserttitle \par}
    {\usebeamerfont{subtitle}\usebeamercolor[fg]{subtitle}\insertsubtitle \par}
    \vspace{1.5em}
    \begin{flushright}
        \usebeamerfont{author}\insertauthor\par
        \usebeamerfont{institute}\insertinstitute \par
        \vspace{3em}
        \usebeamerfont{date}\insertdate\par
        \usebeamercolor[fg]{titlegraphic}\inserttitlegraphic
    \end{flushright}
}
\fi

%%%%%%%%%%%%%%%%%%%%%%%%%%%
%%% how to typset code
%%%%%%%%%%%%%%%%%%%%%%%%%%%

\lstset{language = C,
numbers = left,
numberstyle = {\tiny \emph},
numbersep = 10pt,
breaklines = true,
breakindent = 40pt,
frame = tlRB,
frameround = ffft,
framesep = 3pt,
rulesep = 1pt,
rulecolor = {\color{blue}},
rulesepcolor = {\color{blue}},
flexiblecolumns = true,
keepspaces = true,
basicstyle = \ttfamily\scriptsize,
identifierstyle = ,
commentstyle = ,
stringstyle = ,
showstringspaces = false,
tabsize = 4,
escapechar=\@,
}

\AtBeginSection[]
{
\begin{frame}
\frametitle{}
\tableofcontents[currentsection]
\end{frame}
}

\AtBeginSubsection[]
{
\begin{frame}
\frametitle{}
\tableofcontents[currentsection,currentsubsection]
\end{frame}
}


\title{プロセス}
\begin{document}
\maketitle

%%%%%%%%%%%%%%%%%%%%%%%%%%%%%%%%%% 
% \begin{frame}
% \frametitle{目次}
% \tableofcontents
% \end{frame}

% %%%%%%%%%%%%%%%%% %%%%%%%%%%%%%%%%%
% \section{プロセス}
% %%%%%%%%%%%%%%%%% %%%%%%%%%%%%%%%%%

%%%%%%%%%%%%%%%%% 
\begin{frame}
\frametitle{用語の整理}
\begin{itemize}
\item \ao{プログラム} $=$ 実行すべき命令が書かれているもの
  \begin{itemize}
  \item Firefox, シェル(bash), ls, a.out, など
  \item 実体としてはファイルとして存在している
  \end{itemize}
\item \ao{\bf プロセス} $\approx$ \ao{プログラム}が走っているもの
  \begin{itemize}
  \item メニューでアプリを起動するアイコンをクリックしたり,
    コマンドプロンプトにlsと打ち込むたびに, プロセスが作られている
  \end{itemize}
\item たとえ話
  \begin{itemize}
  \item \ao{プログラム} $\approx$ マニュアル
  \item \ao{プロセス} $\approx$ マニュアルに従って働いている人
  \end{itemize}
\end{itemize}
\end{frame}

%%%%%%%%%%%%%%%%% 
\begin{frame}
\frametitle{プロセスの役割}
\begin{itemize}
\item CPUを分け合うための抽象化
  \begin{itemize}
  \item ユーザはプロセスを作る
  \item 各プロセスは全力で走れば良い(他の人にCPUを譲る必要ない)
  \item OSがプロセスにCPUを与えたり奪ったりする
  \end{itemize}
\item メモリを分け合うための抽象化(アドレス空間)
  \begin{itemize}
  \item 他のプロセスのメモリは覗けない, 壊せない
  \end{itemize}
\end{itemize}
いずれも仕組みは後の週で
\end{frame}

%%%%%%%%%%%%%%%%% 
\begin{frame}
\frametitle{プロセスを観察するコマンド}
\begin{itemize}
\item Unix CUI
  \begin{itemize}
  \item \ao{\tt ps} : 基本
  \item \ao{\tt top} : 時々刻々表示
  \item \ao{\tt pstree} : プロセスの親子関係も表示
  \item \ao{\tt pgrep} : 色々な基準でプロセスを検索
    ($\approx$ {\tt ps} $+$ {\tt grep})
  \end{itemize}
\item Linux
  \begin{itemize}
  \item \ao{\tt /proc/{\it pid}}
  \end{itemize}
\item Ubuntu GUI
  \begin{itemize}
  \item \ao{システムモニタ}
  \end{itemize}
\item Windows GUI
  \begin{itemize}
  \item \ao{タスクマネージャ}
  \item \ao{リソースマネージャ}
  \end{itemize}
\end{itemize}

注:
\begin{itemize}
\item CUI (Character User Interface) 端末の中で字だけ出す
\item GUI (Graphical User Interface) 窓を出して絵を出す
\end{itemize}
\end{frame}

%%%%%%%%%%%%%%%%% 
\begin{frame}[fragile]
  \frametitle{{\tt ps auxww}}
  \begin{itemize}
  \item すべてのプロセスをコマンドライン含め表示
\begin{lstlisting}
$ @{\tt\underline{ps auxww}}@
\end{lstlisting}%$
\item ps と grepを組み合わせてトラブルシューティングする場面がよくある
\begin{lstlisting}
$ @{\tt\underline{ps auxww | grep ssh}}@  # ssh走ってるか?
$ @{\tt\underline{ps auxww | grep tau}}@  # ユーザtauのプロセス
\end{lstlisting}
\item \ao{\tt pgrep} $\approx$ {\tt ps} $+$ {\tt grep}
\begin{lstlisting}
$ @{\tt\underline{pgrep -f ssh}}@
$ @{\tt\underline{pgrep -f tau}}@
\end{lstlisting}
\end{itemize}

表記上の約束: 
\begin{itemize}
\item {\tt \$}はコマンドプロンプトのつもり
\item {\tt \$}以降(下線)が入力すべきもの
\end{itemize}
\end{frame}

%%%%%%%%%%%%%%%%% 
\begin{frame}[fragile]
  \frametitle{プロセスID (PID)}
  \begin{itemize}
  \item 存在しているプロセス全てに付けられている一意な識別子
  \item Linuxでは通常, 4194304までの整数
  \item あるプロセスのPIDを知らなくてはいけない場面はそう多くはないが,
    外から強制終了(kill)したい場合などに必要
  \end{itemize}
\end{frame}

%%%%%%%%%%%%%%%%% 
\begin{frame}[fragile]
  \frametitle{{\tt man}コマンド}
  \begin{itemize}
  \item 以降, コマンドや関数, システムコールの説明が出てきたら,
    必要に応じて{\tt man}コマンドで調べよ
  \item 例:
\begin{lstlisting}
$ @{\tt\underline{man ps}}@
$ @{\tt\underline{man top}}@
$ @{\tt\underline{man pstree}}@
\end{lstlisting} %$
  \end{itemize}
\end{frame}

%%%%%%%%%%%%%%%%% 
\begin{frame}
\frametitle{Unix: プロセス関連のシステムコール}
\begin{itemize}
\item \ao{\tt fork}
  \begin{itemize}
  \item プロセスを作る(コピーする)
  \end{itemize}
\item \ao{\tt execve} 
  \begin{itemize}
  \item 現プロセスで指定のプログラムを実行する
  \item 変種: {\tt exec\{v,l\}p?e?} (引数の渡し方, 微妙な意味の違い)
  \item 以下, 総称してexecと呼ぶ(実際にはexecという名前の関数はない)
  \end{itemize}
\item \ao{\tt exit}
  \begin{itemize}
  \item 現プロセスを終了する
  \item {\tt \_exit}
  \end{itemize}
\item \ao{\tt waitpid}
  \begin{itemize}
  \item 子プロセスの終了待ち $+$ 処理
  \item 変種: {\tt wait, wait3, wait4}
  \end{itemize}
\end{itemize}
\end{frame}
  
%%%%%%%%%%%%%%%%% 
\begin{frame}
\frametitle{子プロセスの生成〜終了〜処理}
\begin{columns}
  \begin{column}{0.5\textwidth}
    \begin{enumerate}
    \item<3-> fork 〜 プロセスが複製される. 親と子が両方, forkの続きを実行
    \item<4-> 子プロセスが終了する(exitを呼ぶ, main関数がreturnするなど)
    \item<5-> 親が処理する(wait, waitpidなどを呼ぶ)まで,
      子プロセスは\ao{「ゾンビ(プロセス番号だけが存在する)状態」}
    \item<6-> 親が処理を終えるとすべてがなくなる
    \end{enumerate}
  \end{column}
  \begin{column}{0.5\textwidth}
\begin{center}
\only<1>{\includegraphics[height=0.7\textheight]{out/pdf/svg/process_1.pdf}}%
\only<2>{\includegraphics[height=0.7\textheight]{out/pdf/svg/process_2.pdf}}%
\only<3>{\includegraphics[height=0.7\textheight]{out/pdf/svg/process_3.pdf}}%
\only<4>{\includegraphics[height=0.7\textheight]{out/pdf/svg/process_4.pdf}}%
\only<5>{\includegraphics[height=0.7\textheight]{out/pdf/svg/process_5.pdf}}%
\only<6>{\includegraphics[height=0.7\textheight]{out/pdf/svg/process_6.pdf}}
\end{center}
  \end{column}
\end{columns}
\end{frame}

%%%%%%%%%%%%%%%%% 
\begin{frame}[fragile]
\frametitle{{\tt fork}}
\begin{itemize}
\item 呼び出したプロセスを複製
\item {\tt fork()}続きが2プロセス(親と子)で実行される
\item 親と子で返り値だけが違う
  \begin{itemize}
  \item 親: 子プロセスのプロセス番号
  \item 子: 0
  \end{itemize}
\item 従って以下がテンプレート
\begin{lstlisting}
pid_t pid = fork();
if (pid == -1) {
  失敗 (子プロセスは作られていない)
} else if (pid == 0) {          /* child */
  子プロセス
} else {          /* child */
  親プロセス
}
\end{lstlisting}
\item 注: {\tt pid\_t} はプロセス番号(process ID)の型; 実際は単なる整数
\end{itemize}
\end{frame}

%%%%%%%%%%%%%%%%% 
\begin{frame}[fragile]
\frametitle{{\tt exit}}
\begin{itemize}
\item {\tt exit}を呼んだプロセスを,
  指定した終了ステータス(exit status)で終了させる
\item exit status : 0 .. 255の整数
\begin{lstlisting}
exit(status);
  ...
\end{lstlisting}
当然ながらexit呼び出し以降(上記の{\tt ...})は実行されない
\item {\tt main}関数が終了した場合も同じ効果
  (従って{\tt main}関数の最後にわざわざ呼ばないのが普通)
  \begin{itemize}
  \item mainの返り値(return value)がexit status
  \end{itemize}
\item exit statusは親プロセスが取得可能
\end{itemize}
\end{frame}

%%%%%%%%%%%%%%%%% 
\begin{frame}[fragile]
  \frametitle{{\tt waitpid}}
  \begin{itemize}
  \item 基本は子プロセスの終了待ち$+$処理(ゾンビ状態を解消; プロセス番号の回収)
  \item どの子プロセス(特定プロセス, どれでもよい, など)を対象とするか,
    終了を待つ・待たない, などを指定可
\begin{lstlisting}
int ws;
pid_t pid = -1;  /* -1 : どの子プロセスでも... */
int options = 0; /* 0 : 終了するまで待つ */
pid_t cid = waitpid(pid, &ws, options);
if (cid == -1) {
  失敗;
} else {
  ... wsに, 子プロセスcidに何が起きたかの情報 ...
}  
\end{lstlisting}
\begin{itemize}
\item 普通の終了: exitを呼んだ / mainがreturnした
\item 異常終了: segmentation faultなど. 詳細は後の週で
\item 停止, 再開 : 詳細略
\end{itemize}
\item 詳細は(これからいつも)manを参照
\begin{lstlisting}
$ @{\tt\underline{man waitpid}}@
\end{lstlisting} %$
\end{itemize}
\end{frame}

%%%%%%%%%%%%%%%%% 
\begin{frame}[fragile]
  \frametitle{manコマンドの落とし穴}
  \begin{itemize}
  \item manコマンドでは, 色々なものが検索できる
    \begin{itemize}
    \item コマンド, システムコール, それ以外の関数(ライブラリ)など
    \end{itemize}
  \item 同じ名前の, コマンドと関数がある場合,
    目当てでないほうが見つかってしまう場合がある
  \item システムコールだと思ったものがライブラリ関数ということもよくある
  \item {\tt -s}オプションで, 種類(正確にはマニュアルの「節」)を指定可能
    \begin{itemize}
    \item \ao{\tt -s 1} : コマンド
    \item \ao{\tt -s 2} : システムコール
    \item \ao{\tt -s 3} : ライブラリ関数
    \end{itemize}
  \item 例:
\begin{lstlisting}
$ man -s 1 wait    # waitコマンド
$ man -s 2 wait    # waitシステムコール
$ man -s 2 execv   # 見つからない
$ man -s 3 execv
\end{lstlisting}
  \end{itemize}
\end{frame}

%%%%%%%%%%%%%%%%% 
\begin{frame}
\frametitle{ゾンビ (defunct)}
\begin{itemize}
\item プロセス$C$が\ao{ゾンビ (defunct)} $\equiv$
  $C$が終了しているが, その親が({\tt waitpid}などで)$C$の終了を確認していない
\item 本質的には, \ao{$C$のプロセス番号を再利用できない}状態
  \begin{itemize}
  \item {\tt waitpid}が「$C$が終了した」と親に知らせるまでは,
    $C$のプロセス番号を他のプロセスに再利用すると,
    プロセス番号からプロセスを一意に特定できなくなるので
  \end{itemize}
\item {\tt waitpid} $\approx$ \ao{お葬式};
  子プロセスに「成仏」「輪廻転生」してもらう
\item 英語では, ``the parent {\it reaps} the child'' という
\end{itemize}
\end{frame}

%%%%%%%%%%%%%%%%% 
\begin{frame}
  \frametitle{{\small 細かい注} {\tt waitpid}, ゾンビにまつわる注}
  \begin{itemize}
  \item Q. 子が終了する前に親が{\tt waitpid}等を呼んだら?
    \begin{itemize}
    \item A. 子が終了するまでreturnしない(waitという名の通り)
    \end{itemize}
  \item Q. 子が終了する前に親が終了できる?
    \begin{itemize}
    \item A. できる
    \end{itemize}
  \item Q. その場合, 誰がその子の葬式をする(その子はゾンビ状態のまま)?
    \begin{itemize}
    \item A. あるプロセス$C$の親が, $C$より先に終了したら,
      全プロセスの先祖 (init) が$C$の親をすることになっている
    \end{itemize}
  \end{itemize}
\end{frame}

%%%%%%%%%%%%%%%%% 
\begin{frame}
\frametitle{子プロセスの生成〜exec}
\begin{columns}
  \begin{column}{0.5\textwidth}
    \begin{enumerate}
    \item<3-> fork 〜 プロセスが複製される. 親と子が両方, forkの続きを実行
    \item<4-> 子プロセスがexecを実行
    \end{enumerate}
  \end{column}
  \begin{column}{0.5\textwidth}
\begin{center}
\only<1>{\includegraphics[height=0.7\textheight]{out/pdf/svg/process_1.pdf}}%
\only<2>{\includegraphics[height=0.7\textheight]{out/pdf/svg/process_2.pdf}}%
\only<3>{\includegraphics[height=0.7\textheight]{out/pdf/svg/process_3.pdf}}%
\only<4>{\includegraphics[height=0.7\textheight]{out/pdf/svg/process_7.pdf}}%
\only<5>{\includegraphics[height=0.7\textheight]{out/pdf/svg/process_8.pdf}}%
\end{center}
  \end{column}
\end{columns}
\end{frame}

%%%%%%%%%%%%%%%%% 
\begin{frame}[fragile]
  \frametitle{exec}
  \begin{itemize}
  \item \ao{現プロセスで}, 指定したプログラムを実行する
  \item 以下は{\tt /bin/ls} (いわゆる{\tt ls}コマンド)を{\tt -l}オプションで実行
\begin{lstlisting}
char * const argv[] = { "/bin/ls", "-l", 0 };
execv(argv[0], argv);
  ...
\end{lstlisting}
\item 注:
  \begin{itemize}
  \item execを呼び出した続き(上記{\tt ...}以降)は実行\aka{されない}
  \item 呼び出したプロセスが, 今持っている状態をすべて忘れ,
    指定したプログラムを実行するだけの人に\ao{化ける}
  \item 特に, execが子プロセスを作るわけではない
  \end{itemize}
\end{itemize}
\end{frame}

%%%%%%%%%%%%%%%%% 
\begin{frame}[fragile]
  \frametitle{{\small (細かい注)} execの変種}
  \begin{itemize}
  \item {\tt exec\{v,l\}p?e?}
  \item ただし {\tt execlpe} は存在しない
  \item つまり {\tt execv, \ao{execve}, execvp, execvpe, execl, execle, execlp}
  \item \ao{\tt execve}だけがシステムコール, 残りはそれの亜種
  \item {\tt v}と{\tt l} : 引数の渡し方({\tt v} : 配列; {\tt l} : 引数のリスト)
  \item {\tt p} : 環境変数{\tt PATH}を参照してコマンドを検索する
    \begin{lstlisting}
execv("ls", argv);       // NG
execv("/bin/ls", argv);  // OK
execvp("ls", argv);      // OK
    \end{lstlisting}
  \item {\tt e} : 子プロセスの環境変数を指定する(ない場合は親を引き継ぐ)
  \end{itemize}
\end{frame}

\end{document}
