\documentclass[12pt,dvipdfmx]{beamer}
\institute{}
\author{田浦健次朗}
\date{}

\usepackage{graphicx}
\DeclareGraphicsExtensions{.pdf}
\DeclareGraphicsExtensions{.eps}
\graphicspath{{out/}{out/tex/}{out/tex/gpl/}{out/tex/svg/}{out/tex/lsvg/}{out/tex/dot/}}
% \graphicspath{{out/}{out/tex/}{out/pdf/}{out/eps/}{out/tex/gpl/}{out/tex/svg/}{out/pdf/dot/}{out/pdf/gpl/}{out/pdf/img/}{out/pdf/odg/}{out/pdf/svg/}{out/eps/dot/}{out/eps/gpl/}{out/eps/img/}{out/eps/odg/}{out/eps/svg/}}
\usepackage{listings,jlisting}
\usepackage{fancybox}
\usepackage{hyperref}
\usepackage{color}

%%%%%%%%%%%%%%%%%%%%%%%%%%%
%%% themes
%%%%%%%%%%%%%%%%%%%%%%%%%%%
\usetheme{default} % Szeged
%% no navigation bar
% default boxes Bergen Boadilla Madrid Pittsburgh Rochester
%% tree-like navigation bar
% Antibes JuanLesPins Montpellier
%% toc sidebar
% Berkeley PaloAlto Goettingen Marburg Hannover Berlin Ilmenau Dresden Darmstadt Frankfurt Singapore Szeged
%% Section and Subsection Tables
% Copenhagen Luebeck Malmoe Warsaw

%%%%%%%%%%%%%%%%%%%%%%%%%%%
%%% innerthemes
%%%%%%%%%%%%%%%%%%%%%%%%%%%
% \useinnertheme{circles}	% default circles rectangles rounded inmargin

%%%%%%%%%%%%%%%%%%%%%%%%%%%
%%% outerthemes
%%%%%%%%%%%%%%%%%%%%%%%%%%%
% outertheme
% \useoutertheme{default}	% default infolines miniframes smoothbars sidebar sprit shadow tree smoothtree


%%%%%%%%%%%%%%%%%%%%%%%%%%%
%%% colorthemes
%%%%%%%%%%%%%%%%%%%%%%%%%%%
\usecolortheme{seahorse}
%% special purpose
% default structure sidebartab 
%% complete 
% albatross beetle crane dove fly seagull 
%% inner
% lily orchid rose
%% outer
% whale seahorse dolphin

%%%%%%%%%%%%%%%%%%%%%%%%%%%
%%% fontthemes
%%%%%%%%%%%%%%%%%%%%%%%%%%%
\usefonttheme{serif}  
% default professionalfonts serif structurebold structureitalicserif structuresmallcapsserif

%%%%%%%%%%%%%%%%%%%%%%%%%%%
%%% generally useful beamer settings
%%%%%%%%%%%%%%%%%%%%%%%%%%%
% 
\AtBeginDvi{\special{pdf:tounicode EUC-UCS2}}
% do not show navigation
\setbeamertemplate{navigation symbols}{}
% show page numbers
\setbeamertemplate{footline}[frame number]

%%%%%%%%%%%%%%%%%%%%%%%%%%%
%%% define some colors for convenience
%%%%%%%%%%%%%%%%%%%%%%%%%%%

\newcommand{\mido}[1]{{\color{green}#1}}
\newcommand{\mura}[1]{{\color{purple}#1}}
\newcommand{\ore}[1]{{\color{orange}#1}}
\newcommand{\ao}[1]{{\color{blue}#1}}
\newcommand{\aka}[1]{{\color{red}#1}}

\setbeamercolor{ex}{bg=cyan!20!white}

\definecolor{UniBlue}{RGB}{20,20,250} 
\setbeamercolor{structure}{fg=UniBlue} % 見出しカラー

\iffalse
%%%%%%%%%%%%%%%%%%%%%%%%%%%
%% customize beamer template
%% https://www.opt.mist.i.u-tokyo.ac.jp/~tasuku/beamer.html
%%%%%%%%%%%%%%%%%%%%%%%%%%%

%\renewcommand{\familydefault}{\sfdefault}  % 英文をサンセリフ体に
%\renewcommand{\kanjifamilydefault}{\gtdefault}  % 日本語をゴシック体に
\usefonttheme{structurebold} % タイトル部を太字
\setbeamerfont{alerted text}{series=\bfseries} % Alertを太字
\setbeamerfont{section in toc}{series=\mdseries} % 目次は太字にしない
\setbeamerfont{frametitle}{size=\Large} % フレームタイトル文字サイズ
\setbeamerfont{title}{size=\LARGE} % タイトル文字サイズ
\setbeamerfont{date}{size=\small}  % 日付文字サイズ

\definecolor{UniBlue}{RGB}{0,150,200} 
\definecolor{AlertOrange}{RGB}{255,76,0}
\definecolor{AlmostBlack}{RGB}{38,38,38}
\setbeamercolor{normal text}{fg=AlmostBlack}  % 本文カラー
\setbeamercolor{structure}{fg=UniBlue} % 見出しカラー
\setbeamercolor{block title}{fg=UniBlue!50!black} % ブロック部分タイトルカラー
\setbeamercolor{alerted text}{fg=AlertOrange} % \alert 文字カラー
\mode<beamer>{
    \definecolor{BackGroundGray}{RGB}{254,254,254}
    \setbeamercolor{background canvas}{bg=BackGroundGray} % スライドモードのみ背景をわずかにグレーにする
}


%フラットデザイン化
\setbeamertemplate{blocks}[rounded] % Blockの影を消す
\useinnertheme{circles} % 箇条書きをシンプルに
\setbeamertemplate{navigation symbols}{} % ナビゲーションシンボルを消す
\setbeamertemplate{footline}[frame number] % フッターはスライド番号のみ

%タイトルページ
\setbeamertemplate{title page}{%
    \vspace{2.5em}
    {\usebeamerfont{title} \usebeamercolor[fg]{title} \inserttitle \par}
    {\usebeamerfont{subtitle}\usebeamercolor[fg]{subtitle}\insertsubtitle \par}
    \vspace{1.5em}
    \begin{flushright}
        \usebeamerfont{author}\insertauthor\par
        \usebeamerfont{institute}\insertinstitute \par
        \vspace{3em}
        \usebeamerfont{date}\insertdate\par
        \usebeamercolor[fg]{titlegraphic}\inserttitlegraphic
    \end{flushright}
}
\fi

%%%%%%%%%%%%%%%%%%%%%%%%%%%
%%% how to typset code
%%%%%%%%%%%%%%%%%%%%%%%%%%%

\lstset{language = C,
numbers = left,
numberstyle = {\tiny \emph},
numbersep = 10pt,
breaklines = true,
breakindent = 40pt,
frame = tlRB,
frameround = ffft,
framesep = 3pt,
rulesep = 1pt,
rulecolor = {\color{blue}},
rulesepcolor = {\color{blue}},
flexiblecolumns = true,
keepspaces = true,
basicstyle = \ttfamily\scriptsize,
identifierstyle = ,
commentstyle = ,
stringstyle = ,
showstringspaces = false,
tabsize = 4,
escapechar=\@,
}

\AtBeginSection[]
{
\begin{frame}
\frametitle{}
\tableofcontents[currentsection]
\end{frame}
}

\AtBeginSubsection[]
{
\begin{frame}
\frametitle{}
\tableofcontents[currentsection,currentsubsection]
\end{frame}
}


\title{ファイルディスクリプタと擬似ファイル \\
副題: Unix的な考え方(全てがファイル)}
\begin{document}
\maketitle

%%%%%%%%%%%%%%%%%%%%%%%%%%%%%%%%%% 
\begin{frame}
\frametitle{目次}
\tableofcontents
\end{frame}

%%%%%%%%%%%%%%%%% 
\section{Unix的なもの}
%%%%%%%%%%%%%%%%% 

%%%%%%%%%%%%%%%%% 
\begin{frame}[fragile]
  \frametitle{Unixの特徴(1) --- 出力先の変更}
  \begin{itemize}
  \item 「端末に出力するプログラム」がそのまま
    「ファイルに出力するプログラム」になる
\begin{lstlisting}
int main() { printf("hi world\n"); }
\end{lstlisting}
\item 普通に走らせると端末へ出力
\begin{lstlisting}
$ ./hello
hi world
\end{lstlisting} %$

\item 出力先を変更(\ao{リダイレクト})する「だけ」でファイルに書ける
\begin{lstlisting}
$ ./hello > hi
\end{lstlisting} %$
  \end{itemize}

ファイルを読み書きするプログラムを書くのに,
ファイルを開く必要がない
\end{frame}

%%%%%%%%%%%%%%%%% 
\begin{frame}[fragile]
  \frametitle{Unixの特徴(2) --- 入力先の変更}
  \begin{itemize}
  \item 同様に「端末から入力するプログラム」がそのまま
    「ファイルから入力するプログラム」になる
\begin{lstlisting}
int main() {
  int x;
  scanf("%d", &x);
  printf("%d\n", x + 1);
  return 0;
}
\end{lstlisting}
\item 端末から
\begin{lstlisting}
$ ./exec/plus_1 
3  # 入力
4  # 出力 (3 + 1)
\end{lstlisting} %$
\item ファイルから
\begin{lstlisting}
$ cat three
3  
$ ./plus_1  < three
4
\end{lstlisting}
\end{itemize}
\end{frame}

%%%%%%%%%%%%%%%%% 
\begin{frame}[fragile]
  \frametitle{Unixの特徴(3) --- パイプでプロセス間通信}
  \begin{itemize}
  \item 「端末に出力するプログラム」がそのまま
    「プロセスにデータを送るプログラム」になる
\begin{lstlisting}
$ echo 10 | ./plus_1 
\end{lstlisting} %$
\item {\tt plus\_1}自身は同様にscanfを呼んでいるだけ
  \item よく使う実例
\begin{lstlisting}
$ ps auxww | grep firefox
\end{lstlisting} %$
\end{itemize}
\end{frame}

%%%%%%%%%%%%%%%%% 
\begin{frame}
  \frametitle{Unixの特徴}
  \begin{itemize}
  \item 同一のプログラムで色々な対象
    \ao{(端末, 普通のファイル, 別のプロセス)}への入出力が可能
  \item そもそもプログラムは今, 読み書きしているものが,
    (普通の)ファイルであるかどうかすら意識せずに書ける
    \begin{itemize}
    \item え? {\tt printf}は端末に書く関数じゃないの?
    \item 否. 「ファイルディスクリプタ1番」に書いている
    \item それが何とつながっているかで出力先が変わる
    \end{itemize}
  \end{itemize}
  それらを可能にした大元の考え方は,
  プロセスの外部とのやり取りは全て
  「ファイルディスクリプタ」を経由して行われるという考え方
\end{frame}


%%%%%%%%%%%%%%%%% 
\begin{frame}
  \frametitle{大元の思想}
  \begin{itemize}
  \item<1-> プロセスのアドレス空間は分離されていて,
    勝手に読み書きできるのはそのプロセスのアドレス空間のみ
  \item<2-> これではプロセスの外とのやり取りが出来ない
  \item<3-> プロセスの外とのやり取りするためにファイルAPIがあった
    \begin{itemize}
    \item fd = open(...);
    \item read(fd, buf, sz);
    \item write(fd, buf, sz);
    \end{itemize}
  \item<4-> プロセスの外との情報の出し入れはなんでもこれ
    ---ファイルディスクリプタにread/writeを発行する---でやる
  \item<5-> ネットワーク, 他のプロセスとの通信, OSからの情報取得, etc.
  \end{itemize}
\end{frame}

%%%%%%%%%%%%%%%%% 
\begin{frame}
  \frametitle{Unix的なもの}
  \begin{itemize}
  \item プロセス外とのやり取りはみな
    「ファイルディスクリプタへのread/write」で統一
  \item さらに, 擬似的なファイル
    (ファイル名などはあるが, 実体は2次記憶と無関係なファイル)
    で多くの機能を提供
  \item ``everything is file''
  \end{itemize}
\end{frame}

%%%%%%%%%%%%%%%%% 
\begin{frame}
  \frametitle{ファイル由来ではないファイルディスクリプタ}
  \begin{itemize}
  \item ソケット
    \begin{itemize}
    \item 目的 : 他のプロセスとの通信 (同一計算機内, ネットワーク越し)
    \item 作り方 : socket システムコール
    \end{itemize}
  \item パイプ
    \begin{itemize}
    \item 目的 : 他のプロセスとの通信 (同一計算機内)
    \item 作り方 : pipe システムコール
    \end{itemize}
  \item シグナルfd (説明しない)
    \begin{itemize}
    \item 目的 : シグナルをOSから受け取る
    \item 作り方 : signalfd システムコール
    \end{itemize}
  \end{itemize}
\end{frame}

%%%%%%%%%%%%%%%%% 
\begin{frame}
  \frametitle{擬似的なファイル}
  \begin{itemize}
  \item 名前付きパイプ(FIFO)
  \item {\tt /proc} ファイルシステム
  \item tmpfs
  \item サウンド, ビデオなどデバイスの入出力
  \end{itemize}
\end{frame}

%%%%%%%%%%%%%%%%% 
\section{リダイレクト, パイプの仕組み}
%%%%%%%%%%%%%%%%%

%%%%%%%%%%%%%%%%% 
\begin{frame}[fragile]
  \frametitle{ファイルディスクリプタの子プロセスへの継承}
  \begin{itemize}
  \item 開いているファイルディスクリプタは,
    fork (プロセス生成) 時に子プロセスへ引き継がれる
    {\footnotesize (親が作ったファイルディスクリプタを, 子プロセスも使える)}
\begin{lstlisting}
int fd = open( ... );
pid_t pid = fork();
if (pid == 0) {
  /* 子プロセス */
  read(fd, buf, sz); /* OK */
} 
\end{lstlisting}
  \item exec後もそのまま有効であり続ける
\begin{lstlisting}
int fd = open("bar", ...);
pid_t pid = fork();
if (pid == 0) {
  /* 子プロセス */
  execv("./foo", ...);
} 
\end{lstlisting}
{\tt foo}の中でも, \mura{もしfdの値がわかれば}, {\tt bar}が読める
\end{itemize}
\end{frame}

%%%%%%%%%%%%%%%%% 
\begin{frame}[fragile]
  \frametitle{標準入出力}
  \begin{itemize}
  \item ほとんどのプログラムは,
  \ao{「0, 1, 2番のディスクリプタが開かれている」}ことを
  前提に書かれている
  \begin{itemize}
  \item 0 : 入力 \ao{(標準入力)}
  \item 1 : 出力 \ao{(標準出力)}
  \item 2 : 出力 \ao{(標準エラー出力)}
  \end{itemize}
\item 入(出)力リダイレクトはfdの値を0 (1)に付け替える
  ($\rightarrow$ {\tt dup2}システムコール)
\end{itemize}
\end{frame}

%%%%%%%%%%%%%%%%% 
\begin{frame}[fragile]
  \frametitle{C言語ストリームAPI}
  \begin{itemize}
  \item C言語ではUnixのopen, read, writeの代わりに,
    以下のAPIを使うことが多い
    \begin{itemize}
    \item {\tt FILE * {\it fp} = fopen({\it filename}, {\it mode});}
    \item {\tt fread({\it buf}, {\it size}, {\it n}, {\it fp});}
    \item {\tt fwrite({\it buf}, {\it size}, {\it n}, {\it fp});}
    \end{itemize}
  \item {\tt FILE} --- \ao{ファイル構造体}
  \item 標準入出力に対応する, {\tt FILE *}型の変数がある
    \begin{itemize}
    \item {\tt stdin} : $\leftrightarrow$ 0
    \item {\tt stdout} : $\leftrightarrow$ 1
    \item {\tt stderr} : $\leftrightarrow$ 2
    \end{itemize}
  \end{itemize}
\end{frame}

%%%%%%%%%%%%%%%%% 
\begin{frame}[fragile]
  \frametitle{高水準なファイル入出力}
  \begin{itemize}
  \item {\tt FILE *}に対しては, より高水準または簡便なAPIもある
    \begin{itemize}
    \item {\tt fgetc({\it fp});} --- 1文字入力
    \item {\tt fgets({\it s}, {\it size}, {\it fp});} --- 1行入力
    \item {\tt fprintf({\it fp}, {\it format}, ...);}
      --- 値を文字列に変換して出力
    \item {\tt fscanf({\it fp}, {\it format}, ...);}
      --- 文字列を値に変換しながら入力
    \end{itemize}
  \item 以下は想像通り
    \begin{itemize}
    \item {\tt getchar()} $\equiv$ {\tt fgetc(stdin);}
    \item {\tt gets({\it s});}
      $\equiv$ {\tt fgets({\it s}, $\infty$, stdin);} \aka{(危険)}
    \item {\tt printf({\it fp}, {\it format}, ...);}
      $\equiv$ {\tt fprintf(stdout, {\it format}, ...);}
    \item {\tt scanf({\it fp}, {\it format}, ...);}
      $\equiv$ {\tt fscanf(stdin, {\it format}, ...);}
    \end{itemize}
  \end{itemize}
\end{frame}

%%%%%%%%%%%%%%%%% 
\begin{frame}[fragile]
  \frametitle{ファイルディスクリプタからファイル構造体}
  \begin{itemize}
  \item (openなどで得た)ファイルディスクリプタに対応した,
    ファイル構造体を作ることが可能
\begin{lstlisting}
FILE * @{\it fp}@ = fdopen(@{\it fd}@, @{\it mode}@);
\end{lstlisting}
\item {\tt FILE *}を得るには{\tt fopen}を使わないといけないわけではない
  \item 使いたいAPIに応じて使い分けることが可能
  \end{itemize}
\end{frame}

%%%%%%%%%%%%%%%%% 
\begin{frame}[fragile]
  \frametitle{{\tt dup2}システムコール}
  \begin{itemize}
  \item {\tt int err = dup2({\it oldfd}, {\it newfd});}
  \item ファイルディスクリプタ {\it oldfd}を{\it newfd}でも使えるようにする
  \item 例えば以下は, ファイル{\tt bar}を0番でも(fdでも)読めるようにする
\begin{lstlisting}
int fd = open("bar", ...);
dup2(fd, 0);      
\end{lstlisting}
  \end{itemize}
\end{frame}
%%%%%%%%%%%%%%%%%

\begin{frame}[fragile]
  \frametitle{入力リダイレクト}
  \begin{columns}
  \begin{column}{0.6\textwidth}
  \begin{itemize}
  \item ``{\it cmd} {\tt <} {\it filename}'' 相当の
    ことをする(シェルのような)プログラム
  \end{itemize}
\begin{lstlisting}
const int fd = open(@{\it filename}@, O_RDONLY);
pid_t pid = fork(); 
if (pid) { /* 親プロセス */
  close(fd); /* 親は不要なfdを閉じる */
} else { /* 子プロセス */
  /* fd -> 0 へ付け替え
  (0を読むと@{\it filename}@が読める) */
  if (fd != 0) {
    dup2(fd, 0);
    close(fd);
  }
  execvp(@{\it cmd}@, ..);
  ... }
\end{lstlisting}
  \end{column}
  \begin{column}{0.4\textwidth}
\begin{center}
\only<1>{\includegraphics[width=\textwidth]{out/pdf/svg/redirect_1.pdf}}%
\only<2>{\includegraphics[width=\textwidth]{out/pdf/svg/redirect_2.pdf}}%
\only<3>{\includegraphics[width=\textwidth]{out/pdf/svg/redirect_3.pdf}}%
\only<4>{\includegraphics[width=\textwidth]{out/pdf/svg/redirect_4.pdf}}%
\only<5>{\includegraphics[width=\textwidth]{out/pdf/svg/redirect_5.pdf}}%
\only<6>{\includegraphics[width=\textwidth]{out/pdf/svg/redirect_6.pdf}}%
\end{center}
  \end{column}
\end{columns}
\end{frame}

%%%%%%%%%%%%%%%%% 

\begin{frame}[fragile]
  \frametitle{出力リダイレクト}
  \begin{itemize}
  \item ``{\it cmd} {\tt >} {\it filename}'' 相当の
    ことをする(シェルのような)プログラム
  \end{itemize}
\begin{lstlisting}
const int fd = creat(@{\it filename}@);
pid_t pid = fork(); 
if (pid) { /* 親プロセス */
  close(fd); /* 親は不要なfdを閉じる */
} else { /* 子プロセス */
  /* fd -> 1 へ付け替え
  (1に書くと@{\it filename}@に書ける) */
  if (fd != 1) {
    close(fd);
    dup2(fd, 1);
  }
  execvp(@{\it cmd}@, ..);
  ... }
\end{lstlisting}
\end{frame}


%%%%%%%%%%%%%%%%% 
\begin{frame}[fragile]
  \frametitle{pipeシステムコール}
  \begin{itemize}
  \item {\tt int rw[2]; int err = pipe(rw);}
    \begin{itemize}
    \item {\tt rw[0], rw[1]}に, それぞれ「読み出し用」「書き込み用」の
      ファイルディスクリプタを書き込み
    \item {\tt rw[1]}に書いたデータが{\tt rw[0]}から読み出せる(パイプ)
    \item もちろん実際の読み書きはread, writeで行える
    \end{itemize}
  \item これと, fork時にファイルディスクリプタが継承する仕組みを使い,
    親子プロセス間での通信が可能
  \end{itemize}
\end{frame}

%%%%%%%%%%%%%%%%% 
\begin{frame}[fragile]
  \frametitle{パイプ (親$\rightarrow$ 子)}
親 $\rightarrow$ 子へデータを送るパターン
  \begin{columns}
    \begin{column}{0.55\textwidth}
\begin{lstlisting}
/* 親がwに書いたものが子の標準入力(0)から読めるようにする */
int rw[2];
pipe(rw);
int r = rw[0], w = rw[1];
pid_t pid = fork();
if (pid) { /* 親プロセス */
  close(r);
  ... w に書き込む ...
  close(w);
} else { /* 子プロセス */
  close(w);
  dup2(r, 0);
  close(r);
  execvp(...); /* 0番から読むコマンド */
}
\end{lstlisting}
\end{column}

\begin{column}{0.45\textwidth}
\begin{center}
\only<1>{\includegraphics[width=\textwidth]{out/pdf/svg/pipe_1.pdf}}%
\only<2>{\includegraphics[width=\textwidth]{out/pdf/svg/pipe_2.pdf}}%
\only<3>{\includegraphics[width=\textwidth]{out/pdf/svg/pipe_3.pdf}}%
\only<4>{\includegraphics[width=\textwidth]{out/pdf/svg/pipe_4.pdf}}%
\only<5>{\includegraphics[width=\textwidth]{out/pdf/svg/pipe_5.pdf}}%
\only<6>{\includegraphics[width=\textwidth]{out/pdf/svg/pipe_6.pdf}}%
\only<7>{\includegraphics[width=\textwidth]{out/pdf/svg/pipe_7.pdf}}%
\end{center}
\end{column}
\end{columns}
\end{frame}

%%%%%%%%%%%%%%%%% 
\begin{frame}[fragile]
  \frametitle{パイプ (子 $\rightarrow$ 親)}
  子 $\rightarrow$ 親へデータを送るパターン
\begin{lstlisting}
/* 子が標準出力(1)に書いたものが親のrから読めるようにする */
int rw[2];
pipe(rw);
int r = rw[0], w = rw[1];
pid_t pid = fork();
if (pid) { /* 親プロセス */
  close(w);
  ... r から読み込む ...
  close(r);
} else { /* 子プロセス */
  close(r);
  dup2(w, 1);
  close(w);
  execvp(...); /* 1番へ書くコマンド */
}
\end{lstlisting}

\begin{itemize}
\item 注: {\tt popen}ライブラリ関数がこれに相当
\end{itemize}

\end{frame}

%%%%%%%%%%%%%%%%% 
\section{擬似ファイル}
%%%%%%%%%%%%%%%%%

%%%%%%%%%%%%%%%%% 
\begin{frame}[fragile]
  \frametitle{擬似ファイル}
  \begin{itemize}
  \item ファイル$=$2次記憶上のデータ, と決めつけるのをやめるのが出発点
  \item openして, read/write 出来るもの(それで有用な動作をするもの)は全て
    「ファイル」にしてしまえ
  \end{itemize}
\end{frame}

%%%%%%%%%%%%%%%%% 
\begin{frame}[fragile]
  \frametitle{名前付きパイプ(FIFO)}
  \begin{itemize}
  \item {\tt int err = mkfifo({\it pathname}, {\it mode});}
  \item 同名のコマンドもある
  \item あるプロセスが書き込んだものが, 読み出すと出てくる
  \item ファイルシステム上に名前を持つ以外, パイプとほぼ同じ機能
  \item コマンド使用例
\begin{lstlisting}
$ mkfifo q
$ cat q   # ブロック
\end{lstlisting}%$
\begin{lstlisting}
$ echo hello > q
\end{lstlisting}%$
  \end{itemize}
\end{frame}

%%%%%%%%%%%%%%%%% 
\begin{frame}[fragile]
  \frametitle{{\tt /proc} ファイルシステム}
  \begin{itemize}
  \item プロセスや, OS内部の情報を読み出し, 変更できるためのファイル群
  \item いろいろ開いてみると良い
    \begin{itemize}
    \item {\tt /proc/cpuinfo} : cpu数, 機種名など
    \item {\tt /proc/meminfo} : メモリサイズや利用状況など
    \item {\tt /proc/{\it pid}/...} : プロセス{\it pid}に関する様々な情報
    \end{itemize}
  \item これらを読むのに普通のファイルを読むコマンド(cat, grep, etc.)が
    使えるのも「Unix的」
  \item これらが実際に2次記憶(HDD?)の中に書かれているわけでは{\bf ない}
  \end{itemize}
\end{frame}

%%%%%%%%%%%%%%%%% 
\begin{frame}[fragile]
  \frametitle{{\tt cgroups} ファイルシステム}
  \begin{itemize}
  \item プロセスの集合に割り当てる資源(CPU, メモリ, etc.)を制御する機能
  \item 使用例
\begin{lstlisting}
sudo mount -t cgroup2 none @{\it dir}@
\end{lstlisting}
\item グループ $\rightarrow$ {\it dir}下のディレクトリで表現
\item 詳しくは{\tt 05\_memory.pdf}のcgroupsの節参照
  \end{itemize}
\end{frame}

%%%%%%%%%%%%%%%%% 
\begin{frame}[fragile]
  \frametitle{tmpfs}
  \begin{itemize}
  \item 実体が(普通の)メモリ上にあるファイルシステム
  \item 再起動時にはデータが失われる
  \item だが, 一部の(少量の)ファイルを高速にアクセスしたい場合には向く
  \item ただしメモリを消費する
  \item ならばOSのキャッシュに任せたほうが良いという説もある
    
  \item 使用例
\begin{lstlisting}
mkdir my_dir
sudo mount -t tmpfs -o size=100M,mode=0755 tmpfs my_dir
sudo chown @{\it user}{\tt :}{\it group}@ my_dir
\end{lstlisting}
  \end{itemize}
  \end{frame}

%%%%%%%%%%%%%%%%% 
\begin{frame}[fragile]
  \frametitle{デバイスファイル}
  \begin{itemize}
  \item 入出力装置(カメラ, マイク, etc.)も,
    あるファイルを読み書きすることで制御やデータの取得が行えるようになっている
  \item 詳細は装置ごとに異なるので深入りしないが,
    これもUnix的な考え方の一例(read, writeして意味がある動作をするものは,
    みなファイルとして見せる)
  \item 単純なデバイスファイル
    \begin{itemize}
    \item {\tt /dev/null} : 書いてもなにもおきない, 読んでもすぐにEOFになる
    \item {\tt /dev/zero} : 書いてもなにもおきない, 読むと無限に0が読み出される
    \item {\tt /dev/urandom} : 乱数(バイナリのバイト列)が読み出される 
    \end{itemize}
  \end{itemize}
\end{frame}

\end{document}
