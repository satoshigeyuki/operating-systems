\documentclass[12pt,dvipdfmx]{beamer}
\institute{}
\author{田浦健次朗}
\date{}

\usepackage{graphicx}
\DeclareGraphicsExtensions{.pdf}
\DeclareGraphicsExtensions{.eps}
\graphicspath{{out/}{out/tex/}{out/tex/gpl/}{out/tex/svg/}{out/tex/lsvg/}{out/tex/dot/}}
% \graphicspath{{out/}{out/tex/}{out/pdf/}{out/eps/}{out/tex/gpl/}{out/tex/svg/}{out/pdf/dot/}{out/pdf/gpl/}{out/pdf/img/}{out/pdf/odg/}{out/pdf/svg/}{out/eps/dot/}{out/eps/gpl/}{out/eps/img/}{out/eps/odg/}{out/eps/svg/}}
\usepackage{listings,jlisting}
\usepackage{fancybox}
\usepackage{hyperref}
\usepackage{color}

%%%%%%%%%%%%%%%%%%%%%%%%%%%
%%% themes
%%%%%%%%%%%%%%%%%%%%%%%%%%%
\usetheme{default} % Szeged
%% no navigation bar
% default boxes Bergen Boadilla Madrid Pittsburgh Rochester
%% tree-like navigation bar
% Antibes JuanLesPins Montpellier
%% toc sidebar
% Berkeley PaloAlto Goettingen Marburg Hannover Berlin Ilmenau Dresden Darmstadt Frankfurt Singapore Szeged
%% Section and Subsection Tables
% Copenhagen Luebeck Malmoe Warsaw

%%%%%%%%%%%%%%%%%%%%%%%%%%%
%%% innerthemes
%%%%%%%%%%%%%%%%%%%%%%%%%%%
% \useinnertheme{circles}	% default circles rectangles rounded inmargin

%%%%%%%%%%%%%%%%%%%%%%%%%%%
%%% outerthemes
%%%%%%%%%%%%%%%%%%%%%%%%%%%
% outertheme
% \useoutertheme{default}	% default infolines miniframes smoothbars sidebar sprit shadow tree smoothtree


%%%%%%%%%%%%%%%%%%%%%%%%%%%
%%% colorthemes
%%%%%%%%%%%%%%%%%%%%%%%%%%%
\usecolortheme{seahorse}
%% special purpose
% default structure sidebartab 
%% complete 
% albatross beetle crane dove fly seagull 
%% inner
% lily orchid rose
%% outer
% whale seahorse dolphin

%%%%%%%%%%%%%%%%%%%%%%%%%%%
%%% fontthemes
%%%%%%%%%%%%%%%%%%%%%%%%%%%
\usefonttheme{serif}  
% default professionalfonts serif structurebold structureitalicserif structuresmallcapsserif

%%%%%%%%%%%%%%%%%%%%%%%%%%%
%%% generally useful beamer settings
%%%%%%%%%%%%%%%%%%%%%%%%%%%
% 
\AtBeginDvi{\special{pdf:tounicode EUC-UCS2}}
% do not show navigation
\setbeamertemplate{navigation symbols}{}
% show page numbers
\setbeamertemplate{footline}[frame number]

%%%%%%%%%%%%%%%%%%%%%%%%%%%
%%% define some colors for convenience
%%%%%%%%%%%%%%%%%%%%%%%%%%%

\newcommand{\mido}[1]{{\color{green}#1}}
\newcommand{\mura}[1]{{\color{purple}#1}}
\newcommand{\ore}[1]{{\color{orange}#1}}
\newcommand{\ao}[1]{{\color{blue}#1}}
\newcommand{\aka}[1]{{\color{red}#1}}

\setbeamercolor{ex}{bg=cyan!20!white}

\definecolor{UniBlue}{RGB}{20,20,250} 
\setbeamercolor{structure}{fg=UniBlue} % 見出しカラー

\iffalse
%%%%%%%%%%%%%%%%%%%%%%%%%%%
%% customize beamer template
%% https://www.opt.mist.i.u-tokyo.ac.jp/~tasuku/beamer.html
%%%%%%%%%%%%%%%%%%%%%%%%%%%

%\renewcommand{\familydefault}{\sfdefault}  % 英文をサンセリフ体に
%\renewcommand{\kanjifamilydefault}{\gtdefault}  % 日本語をゴシック体に
\usefonttheme{structurebold} % タイトル部を太字
\setbeamerfont{alerted text}{series=\bfseries} % Alertを太字
\setbeamerfont{section in toc}{series=\mdseries} % 目次は太字にしない
\setbeamerfont{frametitle}{size=\Large} % フレームタイトル文字サイズ
\setbeamerfont{title}{size=\LARGE} % タイトル文字サイズ
\setbeamerfont{date}{size=\small}  % 日付文字サイズ

\definecolor{UniBlue}{RGB}{0,150,200} 
\definecolor{AlertOrange}{RGB}{255,76,0}
\definecolor{AlmostBlack}{RGB}{38,38,38}
\setbeamercolor{normal text}{fg=AlmostBlack}  % 本文カラー
\setbeamercolor{structure}{fg=UniBlue} % 見出しカラー
\setbeamercolor{block title}{fg=UniBlue!50!black} % ブロック部分タイトルカラー
\setbeamercolor{alerted text}{fg=AlertOrange} % \alert 文字カラー
\mode<beamer>{
    \definecolor{BackGroundGray}{RGB}{254,254,254}
    \setbeamercolor{background canvas}{bg=BackGroundGray} % スライドモードのみ背景をわずかにグレーにする
}


%フラットデザイン化
\setbeamertemplate{blocks}[rounded] % Blockの影を消す
\useinnertheme{circles} % 箇条書きをシンプルに
\setbeamertemplate{navigation symbols}{} % ナビゲーションシンボルを消す
\setbeamertemplate{footline}[frame number] % フッターはスライド番号のみ

%タイトルページ
\setbeamertemplate{title page}{%
    \vspace{2.5em}
    {\usebeamerfont{title} \usebeamercolor[fg]{title} \inserttitle \par}
    {\usebeamerfont{subtitle}\usebeamercolor[fg]{subtitle}\insertsubtitle \par}
    \vspace{1.5em}
    \begin{flushright}
        \usebeamerfont{author}\insertauthor\par
        \usebeamerfont{institute}\insertinstitute \par
        \vspace{3em}
        \usebeamerfont{date}\insertdate\par
        \usebeamercolor[fg]{titlegraphic}\inserttitlegraphic
    \end{flushright}
}
\fi

%%%%%%%%%%%%%%%%%%%%%%%%%%%
%%% how to typset code
%%%%%%%%%%%%%%%%%%%%%%%%%%%

\lstset{language = C,
numbers = left,
numberstyle = {\tiny \emph},
numbersep = 10pt,
breaklines = true,
breakindent = 40pt,
frame = tlRB,
frameround = ffft,
framesep = 3pt,
rulesep = 1pt,
rulecolor = {\color{blue}},
rulesepcolor = {\color{blue}},
flexiblecolumns = true,
keepspaces = true,
basicstyle = \ttfamily\scriptsize,
identifierstyle = ,
commentstyle = ,
stringstyle = ,
showstringspaces = false,
tabsize = 4,
escapechar=\@,
}

\AtBeginSection[]
{
\begin{frame}
\frametitle{}
\tableofcontents[currentsection]
\end{frame}
}

\AtBeginSubsection[]
{
\begin{frame}
\frametitle{}
\tableofcontents[currentsection,currentsubsection]
\end{frame}
}


\title{シグナル}
\begin{document}
\maketitle

%%%%%%%%%%%%%%%%%%%%%%%%%%%%%%%%%%
\iffalse
\begin{frame}
\frametitle{目次}
\tableofcontents
\end{frame}
\fi

%%%%%%%%%%%%%%%%% 
%\section{シグナル}
%%%%%%%%%%%%%%%%% 

%%%%%%%%%%%%%%%%% 
\begin{frame}
  \frametitle{シグナルとは}
  \begin{itemize}
  \item イベント通知のためのAPI
  \item イベント通知 $\rightarrow$ シグナルの発生(配達)
    \begin{itemize}
    \item CPUに対する割り込みの「ソフトウェア(プロセス)版」
    \end{itemize}
  \item どんなイベントに対してシグナルが発生する?
    \begin{itemize}
    \item エラー, 例外的事象
      \begin{itemize}
      \item Segmentation Faultも実はその一つ
      \end{itemize}
    \item 時刻の経過など
    \item 明示的な送信(killシステムコール, killコマンド)
    \item etc.
    \end{itemize}
  \end{itemize}
\end{frame}

%%%%%%%%%%%%%%%%% 
\begin{frame}
  \frametitle{シグナルを受け取る方法}
  \begin{itemize}
  \item 発生したシグナルを受け取るいくつかの方法
    \begin{enumerate}
    \item 登録しておいた関数(シグナルハンドラ)が呼ばれる (sigaction)
    \item ファイルディスクリプタにデータが到着する (signalfd)
    \item シグナルの到着を待つ関数 (sigwaitinfo) に返り値が返される
    \end{enumerate}
  \item デフォルトの動作(何も登録されていない場合)はシグナルにより異なり,
    \begin{itemize}
    \item プロセスが強制終了される
    \item 無視される
    \end{itemize}
  \item ブロックする関数 (readなど) はシグナルを受け取るとリターンするものが多い
    (ブロックしていてもシグナルに気付けるように)
  \end{itemize}
\end{frame}

%%%%%%%%%%%%%%%%% 
\begin{frame}
  \frametitle{シグナルの例}
  \begin{itemize}
  \item []
    \ao{青字}が特によく使う・見かけるもの
    \begin{itemize}
    \item \ao{\tt SIGINT} : interrupt (典型的には端末でCtrl-Cを叩くと発生)
    \item {\tt SIGILL} : 不正命令の実行
    \item \ao{\tt SIGSEGV} : 不正なメモリのアクセス(Segmentation Fault)
    \item \ao{\tt SIGTERM} : プロセスの強制終了のためのシグナル(処理可能)
      \begin{itemize}
      \item {\tt kill}コマンドがデフォルトで送信するシグナル
      \end{itemize}
    \item \ao{\tt SIGKILL} :
      プロセスの強制終了のためのシグナル(処理不可能)
      \begin{itemize}
      \item {\tt kill -9/-KILL}が送信するシグナル
      \end{itemize}
    \item {\tt SIGALRM} :
      時間経過によって発生(alarm, setitimerシステムコール)
    \item {\tt SIGXCPU} :
      CPU使用量超過(処理不可能)
    \item {\tt SIGUSER1, 2} :
      明示的送信のための(自由に利用可能な)シグナル
    \end{itemize}
  \end{itemize}
\end{frame}

%%%%%%%%%%%%%%%%% 
\begin{frame}[fragile]
  \frametitle{シグナルの処理方法 (sigaction)}
  \begin{itemize}
  \item {\tt int sigaction({\it sig}, {\it act}, {\it oldact});}
    \begin{itemize}
    \item {\tt struct sigaction *} {\it act};
    \item シグナル{\it sig}を受信した時の動作を{\it act}で指定
    \end{itemize}
  \item {\tt struct sigaction}の中身
\begin{lstlisting}
struct sigaction {
  ...
  /* 呼ばれる関数を指定するフィールド */
  void (*@\ao{\tt sa\_sigaction}@)(int, siginfo_t *, void *); 
  sigset_t @\ao{\tt sa\_mask}@;
  int @\ao{\tt sa\_flags}@;
    ...
};
\end{lstlisting}
  \end{itemize}
\end{frame}

%%%%%%%%%%%%%%%%% 
\begin{frame}[fragile]
\frametitle{sigaction利用のテンプレート}
\begin{lstlisting}
/* シグナルハンドラ
   @{\it sig}@を受け取ったときに行う動作 */
void sigint_action(int sig, siginfo_t * info, void * arg) {
  ...
}


int main() { 
  ...

  /* @{\it sig}@ を受け取ったら sigint_actionを呼ぶように設定 */
  struct sigaction act;
  act.sa_sigaction = sigint_action;
  sigemptyset(&act.sa_mask);
  act.sa_flags = SA_SIGINFO;
  if (sigaction(@{\it sig}@, &act, 0) == -1) err(1, "sigaction");

  ...
}
\end{lstlisting}
\end{frame}

%%%%%%%%%%%%%%%%% 
\begin{frame}
  \frametitle{sigactionが設定されたシグナルが配達されたときの動作}
  \begin{itemize}
  \item あるスレッド(*)で, (ちょうど割り込みが起きたのと似たように)
    指定されたハンドラが実行される
  \item ハンドラが終了すると, 続きが実行される
  \item (*) あるスレッドはシグナルの種類により,
    \begin{itemize}
    \item そのプロセス中の(OSが適当に選んだ)1つのスレッド
    \item シグナルの原因となる命令を実行したスレッド(SIGSEGV, SIGILLなど)
    \end{itemize}
  \end{itemize}
\end{frame}

%%%%%%%%%%%%%%%%% 
\begin{frame}
  \frametitle{Segmentation Faultもシグナルの一つ}
  \begin{itemize}
  \item Segmentation Fault $=$ 不正なメモリアクセス時に発生 
    \begin{itemize}
    \item 割り当てられていないアドレスをアクセスした
    \item 割り当てられてはいるが, 保護属性(読み・書き・実行)で
      許可されていないアクセスが行われた
    \end{itemize}
  \item 関連システムコール
    \begin{itemize}
    \item {\tt mprotect(\ao{\it addr}, \ao{\it len}, \ao{\it prot});}
    \item {\tt mmap(\ao{\it addr}, \ao{\it len}, \ao{\it prot}, {\it flags}, {\it fd}, {\it offs});}
    \item {\it prot} --- {\tt PROT\_READ, PROT\_WRITE, PROT\_EXEC} (それらのbit和)
    \end{itemize}
  \item 例えば{\tt PROT\_WRITE}が設定されていない領域に書き込みを行うと,
    そのスレッドに{\tt SIGSEGV}シグナルが配達される
    
  \item シグナルハンドラを設定していなければ, プロセスが終了する
    (デフォルトのアクション)
  \end{itemize}
\end{frame}

\end{document}
