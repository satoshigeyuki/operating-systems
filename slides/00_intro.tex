\documentclass[12pt,dvipdfmx]{beamer}
\institute{}
\author{田浦健次朗}
\date{}

\usepackage{graphicx}
\DeclareGraphicsExtensions{.pdf}
\DeclareGraphicsExtensions{.eps}
\graphicspath{{out/}{out/tex/}{out/tex/gpl/}{out/tex/svg/}{out/tex/lsvg/}{out/tex/dot/}}
% \graphicspath{{out/}{out/tex/}{out/pdf/}{out/eps/}{out/tex/gpl/}{out/tex/svg/}{out/pdf/dot/}{out/pdf/gpl/}{out/pdf/img/}{out/pdf/odg/}{out/pdf/svg/}{out/eps/dot/}{out/eps/gpl/}{out/eps/img/}{out/eps/odg/}{out/eps/svg/}}
\usepackage{listings,jlisting}
\usepackage{fancybox}
\usepackage{hyperref}
\usepackage{color}

%%%%%%%%%%%%%%%%%%%%%%%%%%%
%%% themes
%%%%%%%%%%%%%%%%%%%%%%%%%%%
\usetheme{default} % Szeged
%% no navigation bar
% default boxes Bergen Boadilla Madrid Pittsburgh Rochester
%% tree-like navigation bar
% Antibes JuanLesPins Montpellier
%% toc sidebar
% Berkeley PaloAlto Goettingen Marburg Hannover Berlin Ilmenau Dresden Darmstadt Frankfurt Singapore Szeged
%% Section and Subsection Tables
% Copenhagen Luebeck Malmoe Warsaw

%%%%%%%%%%%%%%%%%%%%%%%%%%%
%%% innerthemes
%%%%%%%%%%%%%%%%%%%%%%%%%%%
% \useinnertheme{circles}	% default circles rectangles rounded inmargin

%%%%%%%%%%%%%%%%%%%%%%%%%%%
%%% outerthemes
%%%%%%%%%%%%%%%%%%%%%%%%%%%
% outertheme
% \useoutertheme{default}	% default infolines miniframes smoothbars sidebar sprit shadow tree smoothtree


%%%%%%%%%%%%%%%%%%%%%%%%%%%
%%% colorthemes
%%%%%%%%%%%%%%%%%%%%%%%%%%%
\usecolortheme{seahorse}
%% special purpose
% default structure sidebartab 
%% complete 
% albatross beetle crane dove fly seagull 
%% inner
% lily orchid rose
%% outer
% whale seahorse dolphin

%%%%%%%%%%%%%%%%%%%%%%%%%%%
%%% fontthemes
%%%%%%%%%%%%%%%%%%%%%%%%%%%
\usefonttheme{serif}  
% default professionalfonts serif structurebold structureitalicserif structuresmallcapsserif

%%%%%%%%%%%%%%%%%%%%%%%%%%%
%%% generally useful beamer settings
%%%%%%%%%%%%%%%%%%%%%%%%%%%
% 
\AtBeginDvi{\special{pdf:tounicode EUC-UCS2}}
% do not show navigation
\setbeamertemplate{navigation symbols}{}
% show page numbers
\setbeamertemplate{footline}[frame number]

%%%%%%%%%%%%%%%%%%%%%%%%%%%
%%% define some colors for convenience
%%%%%%%%%%%%%%%%%%%%%%%%%%%

\newcommand{\mido}[1]{{\color{green}#1}}
\newcommand{\mura}[1]{{\color{purple}#1}}
\newcommand{\ore}[1]{{\color{orange}#1}}
\newcommand{\ao}[1]{{\color{blue}#1}}
\newcommand{\aka}[1]{{\color{red}#1}}

\setbeamercolor{ex}{bg=cyan!20!white}

\definecolor{UniBlue}{RGB}{20,20,250} 
\setbeamercolor{structure}{fg=UniBlue} % 見出しカラー

\iffalse
%%%%%%%%%%%%%%%%%%%%%%%%%%%
%% customize beamer template
%% https://www.opt.mist.i.u-tokyo.ac.jp/~tasuku/beamer.html
%%%%%%%%%%%%%%%%%%%%%%%%%%%

%\renewcommand{\familydefault}{\sfdefault}  % 英文をサンセリフ体に
%\renewcommand{\kanjifamilydefault}{\gtdefault}  % 日本語をゴシック体に
\usefonttheme{structurebold} % タイトル部を太字
\setbeamerfont{alerted text}{series=\bfseries} % Alertを太字
\setbeamerfont{section in toc}{series=\mdseries} % 目次は太字にしない
\setbeamerfont{frametitle}{size=\Large} % フレームタイトル文字サイズ
\setbeamerfont{title}{size=\LARGE} % タイトル文字サイズ
\setbeamerfont{date}{size=\small}  % 日付文字サイズ

\definecolor{UniBlue}{RGB}{0,150,200} 
\definecolor{AlertOrange}{RGB}{255,76,0}
\definecolor{AlmostBlack}{RGB}{38,38,38}
\setbeamercolor{normal text}{fg=AlmostBlack}  % 本文カラー
\setbeamercolor{structure}{fg=UniBlue} % 見出しカラー
\setbeamercolor{block title}{fg=UniBlue!50!black} % ブロック部分タイトルカラー
\setbeamercolor{alerted text}{fg=AlertOrange} % \alert 文字カラー
\mode<beamer>{
    \definecolor{BackGroundGray}{RGB}{254,254,254}
    \setbeamercolor{background canvas}{bg=BackGroundGray} % スライドモードのみ背景をわずかにグレーにする
}


%フラットデザイン化
\setbeamertemplate{blocks}[rounded] % Blockの影を消す
\useinnertheme{circles} % 箇条書きをシンプルに
\setbeamertemplate{navigation symbols}{} % ナビゲーションシンボルを消す
\setbeamertemplate{footline}[frame number] % フッターはスライド番号のみ

%タイトルページ
\setbeamertemplate{title page}{%
    \vspace{2.5em}
    {\usebeamerfont{title} \usebeamercolor[fg]{title} \inserttitle \par}
    {\usebeamerfont{subtitle}\usebeamercolor[fg]{subtitle}\insertsubtitle \par}
    \vspace{1.5em}
    \begin{flushright}
        \usebeamerfont{author}\insertauthor\par
        \usebeamerfont{institute}\insertinstitute \par
        \vspace{3em}
        \usebeamerfont{date}\insertdate\par
        \usebeamercolor[fg]{titlegraphic}\inserttitlegraphic
    \end{flushright}
}
\fi

%%%%%%%%%%%%%%%%%%%%%%%%%%%
%%% how to typset code
%%%%%%%%%%%%%%%%%%%%%%%%%%%

\lstset{language = C,
numbers = left,
numberstyle = {\tiny \emph},
numbersep = 10pt,
breaklines = true,
breakindent = 40pt,
frame = tlRB,
frameround = ffft,
framesep = 3pt,
rulesep = 1pt,
rulecolor = {\color{blue}},
rulesepcolor = {\color{blue}},
flexiblecolumns = true,
keepspaces = true,
basicstyle = \ttfamily\scriptsize,
identifierstyle = ,
commentstyle = ,
stringstyle = ,
showstringspaces = false,
tabsize = 4,
escapechar=\@,
}

\AtBeginSection[]
{
\begin{frame}
\frametitle{}
\tableofcontents[currentsection]
\end{frame}
}

\AtBeginSubsection[]
{
\begin{frame}
\frametitle{}
\tableofcontents[currentsection,currentsubsection]
\end{frame}
}


\title{オペレーティングシステム --- イントロ}

\begin{document}
\maketitle

%%%%%%%%%%%%%%%%%%%%%%%%%%%%%%%%%% 
% \begin{frame}
% \frametitle{目次}
% \tableofcontents
% \end{frame}

%%%%%%%%%%%%%%%%% %%%%%%%%%%%%%%%%%
\section{}
%%%%%%%%%%%%%%%%% %%%%%%%%%%%%%%%%%

%%%%%%%%%%%%%%%%% 
\begin{frame}
\frametitle{オペレーティングシステム (Operating System; OS) とは}
\begin{itemize}
\item 実例: Windows, MacOS, Linux, BSD, iOS, Android, etc.
\item アプリケーションを動かすためのソフト(基本ソフト)
\item 存在理由(一般的な言葉で):
  \begin{itemize}
  \item \ao{抽象化:} 簡単にプログラムできるようにする
  \item \ao{効率化:} 簡単なプログラムで高速に動作するようにする
  \item \ao{資源保護・管理:} 資源(CPU, メモリ, etc.)の独占を防ぎ,
    公平に割り当てる
  \end{itemize}
\item 具体的には\mura{OSがないとどうなるか}を知る・考えるのが良い
\end{itemize}
\end{frame}

%%%%%%%%%%%%%%%%% 
\begin{frame}
  \frametitle{OSがないと\ldots}
  \begin{itemize}
  \item CPU (プロセッサ) 上に直接ユーザのプログラムが動く
  \item 例えば以下のようなことが非常に困難になる
    \begin{enumerate}
    \item \ao{CPU} (計算のための資源) を公平に分け合う
    \item \ao{メモリ} (記憶のための資源) を安全に分け合う
    \item \ao{外部ストレージ}を安全に分け合う
    \item \ao{入出力}
    \end{enumerate}
    etc.
  \end{itemize}
\end{frame}

%%%%%%%%%%%%%%%%% 
\begin{frame}
\frametitle{OSがない場合の問題点とOSの機能}
\begin{itemize}
\item CPUを分け合う
  \begin{itemize}
  \item OSなし: 1つのプログラムでCPUを独占できてしまう
  \item OS: \ao{プロセス, スレッド}
  \end{itemize}
\item メモリを分け合う
  \begin{itemize}
  \item OSなし: 1つのプログラムが他の人の
    (メモリ上の)データをのぞき見・破壊できてしまう;
    大量のメモリを独占できてしまう
  \item OS: \ao{プロセス(アドレス空間), 仮想記憶}
  \end{itemize}
\item ストレージを分け合う
  \begin{itemize}
  \item OSなし: 1つのプログラムが他の人のデータをのぞき見・破壊できてしまう
  \item OS: \ao{ファイルシステム}, \ao{システムコール}
  \end{itemize}
\item 入出力
  \begin{itemize}
  \item OSなし: 入力監視, 割り込み処理など複雑, かつ機器依存
  \item OS: \ao{ファイルシステム, プロセス間通信}
  \end{itemize}
\end{itemize}
\end{frame}


%%%%%%%%%%%%%%%%% 
\begin{frame}
  \frametitle{資源保護・管理のための基本的仕組み}
  \begin{itemize}
  \item 命題: 資源(CPU, メモリ, etc.)の独占を防ぎ, 公平に割り当てる
  \item 悪意のあるプログラムでも
    \mura{OSの破壊, 他のプログラムの破壊, 資源の独占}を不可能にする
  \item 以降ではまず\mura{OSの破壊}を不可能にする仕組みを考える
    \begin{itemize}
    \item それ以外は次週以降の個別の機能説明にて
    \end{itemize}
  \end{itemize}
\end{frame}

%%%%%%%%%%%%%%%%% 
\begin{frame}
\frametitle{OSは実体としてはどこにどう存在している?}
  \begin{columns}
    \begin{column}{0.6\textwidth}
      \begin{itemize}
      \item その他のプログラムと同様, メモリ上にプログラム$+$データとして存在
      \item ただし\ao{OS以外のプログラムには読み書き不能}になっている
      \item $\rightarrow$ どうやって?
      \end{itemize}
    \end{column}
    \begin{column}{0.4\textwidth}
      \includegraphics[width=\textwidth]{out/pdf/svg/os_1.pdf}
    \end{column}
  \end{columns}
\end{frame}

%%%%%%%%%%%%%%%%% 
\begin{frame}
\frametitle{CPUの特権モード・ユーザモード}
\begin{itemize}
\item CPUの動作モードに(少なくとも)2種類ある
  \begin{itemize}
  \item \ao{ユーザモード}
  \item \ao{特権モード (スーパバイザモード)}
  \end{itemize}
\end{itemize}
\begin{columns}
  \begin{column}{0.6\textwidth}
    \begin{itemize}
    \item 両者の主な違い
      \begin{enumerate}
      \item 一部の命令が特権モードでしか実行できない(\ao{特権命令})
      \item 一部のメモリ領域に「ユーザモードでアクセス不可」という属性をつけられる
      \end{enumerate}
    \item OSのデータやプログラムが
      \ao{OS以外のプログラムには読み書き不能}な仕組み
      \begin{itemize}
      \item OSが管理する領域を「ユーザモードでアクセス不可」
      \item OS以外はユーザモードで動作
      \end{itemize}
    \end{itemize}
  \end{column}
  \begin{column}{0.4\textwidth}
    \includegraphics[width=\textwidth]{out/pdf/svg/os_1.pdf}
  \end{column}
\end{columns}
\end{frame}

%%%%%%%%%%%%%%%%% 
\begin{frame}
\frametitle{ユーザモードから特権モードへの移行}
\begin{itemize}
\item<1-> 通常のプログラムはユーザモードで実行される
\item<1-> 一方通常のプログラムもOSの機能を呼び出すことが出来る
  (さもなければOSはいらないはず)
\item<2-> $\rightarrow$ ユーザモードから特権モードへ移行する仕組みがあるはず
\item<3-> 下手に設計すれば,
  結局誰でも特権モードで好きな命令を実行可能になる危険
\item<4-> $\rightarrow$ \ao{トラップ命令}
\end{itemize}
\end{frame}

%%%%%%%%%%%%%%%%% 
\begin{frame}
\frametitle{トラップ命令}
\begin{columns}
  \begin{column}{0.6\textwidth}
    \begin{itemize}
    \item 以下の二つを行う
      \begin{itemize}
      \item \ao{ユーザモードから特権モードへ移行}
      \item ある\ao{定められた番地}にジャンプ
      \end{itemize}
    \item x86の場合
      \begin{itemize}
      \item \href{https://www.felixcloutier.com/x86/syscall}{int 0x80h}命令
      \item \href{https://www.felixcloutier.com/x86/syscall}{syscall}命令
      \end{itemize}
    \item<2-> ある\ao{定められた番地}は,
      「割り込みベクタ」と呼ばれるメモリ上の
      配列にかかれており, OSが起動時に設定する
    \item<4-> ユーザプログラムからOSへの「入り口」
      $\rightarrow$ \ao{システムコール}
    \end{itemize}
  \end{column}
  \begin{column}{0.4\textwidth}
    \only<1>{\includegraphics[width=\textwidth]{out/pdf/svg/os_2.pdf}}%
    \only<2>{\includegraphics[width=\textwidth]{out/pdf/svg/os_3.pdf}}%
    \only<3>{\includegraphics[width=\textwidth]{out/pdf/svg/os_4.pdf}}%
    \only<4>{\includegraphics[width=\textwidth]{out/pdf/svg/os_5.pdf}}%
  \end{column}
\end{columns}
\end{frame}

%%%%%%%%%%%%%%%%% 
\begin{frame}
\frametitle{システムコール}
\begin{columns}
  \begin{column}{0.6\textwidth}
    \begin{itemize}
    \item OSがユーザに対して提供している(根源的な)機能
      \begin{itemize}
      \item 実例:
        open, write, read, close, fork, exec, wait, exit, socket, send, recv, etc.
      \item 通常Cの関数として説明されているがこれは説明の便宜上 $+$ ユーザの利便性のため
      \end{itemize}
    \item 本当にシステムコールが呼び出されている瞬間は,
      トラップ命令(前スライド)でOS内の命令に突入する瞬間
    \end{itemize}
  \end{column}
  \begin{column}{0.4\textwidth}
    \only<1>{\includegraphics[width=\textwidth]{out/pdf/svg/os_5.pdf}}%
    \only<2>{\includegraphics[width=\textwidth]{out/pdf/svg/os_6.pdf}}%
  \end{column}
\end{columns}
\end{frame}

%%%%%%%%%%%%%%%%% 
\begin{frame}
\frametitle{保護とシステムコール(すべての保護の基礎)}
\begin{columns}
  \begin{column}{0.6\textwidth}
    \begin{itemize}
    \item OS内には無数の機能が命令列として存在しているが,
      ユーザプログラムからの「入り口」
      (特権モードで実行される最初の命令)がひとつしかない
      
    \item その唯一の入り口から分岐してすべての機能ごとのシステムコールが実行されている
      
    \item ユーザプログラムが正規の入り口(システムコール)を通らずに,
      特権モードに移行することはできない
      
    \item OS内部の(特権モードで実行される)プログラムをしっかり書けば,
      OSを保護可能
    \end{itemize}
  \end{column}
  \begin{column}{0.4\textwidth}
    \includegraphics[width=\textwidth]{out/pdf/svg/os_6.pdf}
  \end{column}
\end{columns}
\end{frame}
\end{document}



